% !Mode:: "TeX:UTF-8"

\documentclass[10pt,a4paper]{article}
\usepackage[utf8]{inputenc}
\usepackage{amsmath}
\usepackage{amsfonts}
\usepackage{amssymb}
\usepackage{graphicx}
\usepackage{CJKutf8}

\setlength{\parindent}{14pt}

\begin{document}
\begin{CJK*}{UTF8}{song}
\author{nuomizong}
\title{关于\LaTeX{}中pdf和eps图片的处理}
\date{Aug. 20th, 2015}
\maketitle

最终latex中的图片格式主要就2种 pdf 和 eps。如果要用pdflatex编译,那么自然选择pdf,如果用latex编译,自然用eps。

图片来源主要是matlab和visio,所以这里主要讲如何从这两个软件中得到清晰的,无空白边缘的pdf和eps。

\section{visio完美另存为pdf}
visio2010本来就用另存为pdf的功能,但是直接存的话,很难避免页面的空白边缘。网上的方法大部分都是用Adobe Acrobat裁剪。这个方法感觉很麻烦。

简单实用方法是:

先用visio画好图,然后“设计”--“大小”--"适应绘图"。visio就会自动调整画布来去除空白边缘,此时再另存外pdf即可。 参见图\ref{fig: visio to pdf}。

\begin{figure}[h]
	\centering
	\includegraphics[width=0.5\textwidth]{visioToPdf}
	\caption{visio to pdf} \label{fig: visio to pdf}
\end{figure}

\section{eps转pdf}
matlab里的图可以另存外eps,但是pdflatex不认eps,怎么办,转吧。

我装的是CTEX最新版。里面自带eps转pdf的工具 epstopdf

如果CTEX正常安装的话, 打开cmd,直接cd到eps文件所在目录(假设名为a.eps),执行命令:epstopdf a.eps  就会在当前目录生成a.pdf

\section{pdf转eps}
这个必须借助Adobe acrobat完成了。

用Adobe acrobat pro打开pdf,然后另存为ps,再用gsviewer(CTEX带的软件)打开ps,另存为eps即可。

\end{CJK*}
\end{document}