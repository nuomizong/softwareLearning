% !Mode:: "TeX:UTF-8"	% read in as utf8 file.
\documentclass[10pt,a4paper]{article}

% !Mode:: "TeX:UTF-8"	% read in as utf8 file.
\usepackage{microtype}

% ---- Math packages. ----
\usepackage{amsmath}
\usepackage{amsfonts}
\usepackage{amssymb}
\usepackage{amsthm}	% thm = theorem
% \usepackage{mathtools}
% \usepackage{array}

\usepackage{siunitx} % standard unit
\DeclareSIUnit\rotation{r}

% ---- Figures and Captions. ----
\usepackage{graphicx}


\usepackage{caption}
% captionof command. when inserting graphics without figure environment, captionof can produce the caption.

\usepackage{subcaption} % subfigure environment.

% ---- Tables. ----
\usepackage{booktabs} % three-line tables: toprule, midrule, bottomrule
\usepackage{longtable}
\usepackage{multirow}

% ---- References. ----
\usepackage[square,sort,comma,numbers]{natbib}
%\usepackage{biblatex}

\usepackage{hyperref} % url command

%======================================================
%	Color
%======================================================
\usepackage{color}
\usepackage{colortbl}
\definecolor{bkg}{rgb}{0.95,0.95,0.92}

%======================================================
%	Todo
%======================================================
\usepackage{todonotes}
\newcommand{\TODO}[1]{{\color{red}{[TODO: #1]}}}


\RequirePackage[l2tabu, orthodox]{nag}

%\usepackage{tikz}

\usepackage[width=21.00cm, height=29.70cm, left=2.54cm, right=2.54cm, top=2.54cm, bottom=2.54cm]{geometry}

\usepackage[utf8]{inputenc}

\usepackage{verbatim}
\usepackage{listings} % can print various codes including listings itself and LaTeX. can also use lstinline command.
% ====== set styles for listings. ======
\definecolor{codegreen}{rgb}{0,0.6,0}
\definecolor{codegray}{rgb}{0.5,0.5,0.5}
\definecolor{codepurple}{rgb}{0.58,0,0.82}

\lstdefinestyle{zstyle}{
	backgroundcolor=\color[rgb]{0.95,0.95,0.92},
	commentstyle=\color{codegreen},
	basicstyle=\ttfamily\small,
	keywordstyle=\color{codepurple},
	numberstyle=\tiny\color{codegray},
	numbersep=5pt,
	stringstyle=\color{red},
	showspaces=false,
	showstringspaces=false,
	showtabs=false,
	numbers=left,
	prebreak=\raisebox{0ex}[0ex][0ex]{\ensuremath{\hookleftarrow}},
	captionpos=b,
	frame=single,
	breakatwhitespace=false,
	breaklines=true,
	keepspaces=true,
	tabsize=4,
	%escapeinside,
}

\lstset{style=zstyle}
% ====== set styles for listings. ======

\usepackage{tabularx} % tabularx environment. equivalent lenth?

\usepackage{CJKutf8} % Chinese, Japanese, Korean input with utf-8 encoding. it loads \usepackage[utf8]{inputenc} internally



%\usepackage{fancyvrb} % not familiar with

\usepackage{cleveref}
%% \crefname{ <type> }{ <singular> }{ <plural> }
%% \Cref{ key} capitalize the first letter.
%\crefname{table}{table}{tables} 
%\crefname{figure}{fig.}{figs.}
%\crefname{equation}{eq.}{eqs.}

\usepackage{newverbs} % if listings package doesn't work, use this one to highlight.
\newverbcommand{\cverb}{\color{red}}{} % colored vertb with red
\newverbcommand{\bverb}	% verbatim with gray background
{\begin{lrbox}{\verbbox}}
	{\end{lrbox}\colorbox{gray}{\box\verbbox}}

%======================== separator

\usepackage{txfonts} % piup
\usepackage{xfrac} % more beautiful and standard fractions. \sfrac{1}{2}

%\usepackage{enumerate}

\numberwithin{equation}{section} % number equation with section number.

%\usepackage[numbers,sort]{natbib} % [1,3,2] => [1,2,3]
%\usepackage[numbers,sort & compress]{natbib} % [1,3,2] => [1-3]

%\setlength{\parindent}{0pt}

\usepackage{enumerate} % [i], [ii]...
%% !Mode:: "TeX:UTF-8"	% read in as utf8 file.
\usepackage{microtype}

% ---- Math packages. ----
\usepackage{amsmath}
\usepackage{amsfonts}
\usepackage{amssymb}
\usepackage{amsthm}	% thm = theorem
% \usepackage{mathtools}
% \usepackage{array}

\usepackage{siunitx} % standard unit
\DeclareSIUnit\rotation{r}

% ---- Figures and Captions. ----
\usepackage{graphicx}


\usepackage{caption}
% captionof command. when inserting graphics without figure environment, captionof can produce the caption.

\usepackage{subcaption} % subfigure environment.

% ---- Tables. ----
\usepackage{booktabs} % three-line tables: toprule, midrule, bottomrule
\usepackage{longtable}
\usepackage{multirow}

% ---- References. ----
\usepackage[square,sort,comma,numbers]{natbib}
%\usepackage{biblatex}

\usepackage{hyperref} % url command

%======================================================
%	Color
%======================================================
\usepackage{color}
\usepackage{colortbl}
\definecolor{bkg}{rgb}{0.95,0.95,0.92}

%======================================================
%	Todo
%======================================================
\usepackage{todonotes}
\newcommand{\TODO}[1]{{\color{red}{[TODO: #1]}}}


\RequirePackage[l2tabu, orthodox]{nag}

%\usepackage{tikz}

\usepackage[width=21.00cm, height=29.70cm, left=2.54cm, right=2.54cm, top=2.54cm, bottom=2.54cm]{geometry}

\usepackage[utf8]{inputenc}

\usepackage{verbatim}
\usepackage{listings} % can print various codes including listings itself and LaTeX. can also use lstinline command.
% ====== set styles for listings. ======
\definecolor{codegreen}{rgb}{0,0.6,0}
\definecolor{codegray}{rgb}{0.5,0.5,0.5}
\definecolor{codepurple}{rgb}{0.58,0,0.82}

\lstdefinestyle{zstyle}{
	backgroundcolor=\color[rgb]{0.95,0.95,0.92},
	commentstyle=\color{codegreen},
	basicstyle=\ttfamily\small,
	keywordstyle=\color{codepurple},
	numberstyle=\tiny\color{codegray},
	numbersep=5pt,
	stringstyle=\color{red},
	showspaces=false,
	showstringspaces=false,
	showtabs=false,
	numbers=left,
	prebreak=\raisebox{0ex}[0ex][0ex]{\ensuremath{\hookleftarrow}},
	captionpos=b,
	frame=single,
	breakatwhitespace=false,
	breaklines=true,
	keepspaces=true,
	tabsize=4,
	%escapeinside,
}

\lstset{style=zstyle}
% ====== set styles for listings. ======

\usepackage{tabularx} % tabularx environment. equivalent lenth?

\usepackage{CJKutf8} % Chinese, Japanese, Korean input with utf-8 encoding. it loads \usepackage[utf8]{inputenc} internally



%\usepackage{fancyvrb} % not familiar with

\usepackage{cleveref}
%% \crefname{ <type> }{ <singular> }{ <plural> }
%% \Cref{ key} capitalize the first letter.
%\crefname{table}{table}{tables} 
%\crefname{figure}{fig.}{figs.}
%\crefname{equation}{eq.}{eqs.}

\usepackage{newverbs} % if listings package doesn't work, use this one to highlight.
\newverbcommand{\cverb}{\color{red}}{} % colored vertb with red
\newverbcommand{\bverb}	% verbatim with gray background
{\begin{lrbox}{\verbbox}}
	{\end{lrbox}\colorbox{gray}{\box\verbbox}}

%======================== separator

\usepackage{txfonts} % piup
\usepackage{xfrac} % more beautiful and standard fractions. \sfrac{1}{2}

%\usepackage{enumerate}

\numberwithin{equation}{section} % number equation with section number.

%\usepackage[numbers,sort]{natbib} % [1,3,2] => [1,2,3]
%\usepackage[numbers,sort & compress]{natbib} % [1,3,2] => [1-3]

%\setlength{\parindent}{0pt}

\usepackage{enumerate} % [i], [ii]...


\begin{document}
\author{Anzong Zheng}
\title{Matlab operations}
\date{October 5, 2016}
\maketitle
	
\tableofcontents
\newpage\clearpage\setcounter{page}{1}

\section{Single quote}
To output single quote use:

\begin{lstlisting}[language = matlab]
'' % two quotes
\end{lstlisting}

\section{Call external exe}
call exe and pass parameters to exe in matlab
\begin{lstlisting}[language=Matlab]
dos('a.exe <paras>')
1) argv[0] = "a.exe"
2) argv[1] = "<paras>"
\end{lstlisting}

\section{Exoport utf-8 encoded file}
\begin{lstlisting}[language=matlab]
fid = fopen('a.txt','w','n','UTF-8');

1) 'w' permission
2) 'n' machinefmt
3) 'UTF-8' encodingIn
\end{lstlisting}

\section{colormap}
\begin{lstlisting}[language=matlab]
% the range of tick is [0, 1]. I am damn right!
colorbar('Ticks',[0, 1], 'TickLabels',{'Small','High'})
\end{lstlisting}

\section{uiopen}
\begin{lstlisting}[language=matlab]
if (exist('fullfilename', 'var') == 0)
[filename, filefolder] = uigetfile('.obj', 'Read obj-file');
fullfilename = [filefolder filename];
end
\end{lstlisting}

\section{variable existence}
\begin{lstlisting}[language=matlab]
if (exist('fullfilename', 'var') == 0)
	[filename, filefolder] = uigetfile('.obj', 'Read obj-file');
	fullfilename = [filefolder filename];
end
\end{lstlisting}

\section{Screen output}
disp or fprintf

\section{variable arguments input}
\begin{lstlisting}[language=matlab]
function varlist2(X,Y,varargin)
	fprintf('Total number of inputs = %d\n',nargin);
	
	nVarargs = length(varargin);
	fprintf('Inputs in varargin(%d):\n',nVarargs)
	for k = 1:nVarargs
	fprintf('   %d\n', varargin{k})
end
\end{lstlisting}

\section{Triangular mesh}
Both trisurf and trimesh functions are based on patch function.

\section{Viewport}
\begin{lstlisting}[language=matlab]
view(2) or view(3)
view([90,0])
\end{lstlisting}

\section{Export to specified path}
\begin{lstlisting}[language=matlab]
	% note the folder must exist!
	modelName = 'bridge'; version = '_f16_b29_rib2';
	folder = ['D:\360Sync\ribShell_zong\Codes\models\',modelName,version,'\'];
	subfolder = 'other\';
	Writer_PrinStress([folder,modelName,'.vStress'], vpStress);
	saveas(gcf,[folder,subfolder,modelName],'fig');
\end{lstlisting}

\section{Search path}
\begin{lstlisting}[language=matlab]
	modelName = 'bridge'; version = '_f16_b29_rib2';
	folder = ['D:\360Sync\ribShell_zong\Codes\models\',modelName,version,'\'];
	subfolder = 'other\';
	
	% note the folder must exist!
	addpath(folder);
	...
	rmpath(folder);
\end{lstlisting}

\section{Global}
To clear global variables and release used memory simply use:
\begin{lstlisting}[language=matlab]
clear global;
\end{lstlisting}

\section{Mex functions}
\subsection{Mac SDK support}
Go to \textcolor{red}{<matlab folder>, bin, maci64, mexopts}, open \lstinline{clang++_maci64.xml}.
Add sentences like:

\begin{lstlisting}[language=xml]
<dirExists name="$$/Platforms/MacOSX.platform/Developer/SDKs/MacOSX10.13.sdk" />
<dirExists name="$$/Platforms/MacOSX.platform/Developer/SDKs/MacOSX10.12.sdk" />
...
<cmdReturns name="find $$ -name MacOSX10.12.sdk | egrep 'MacOSX10.13.sdk'" />
\end{lstlisting}

\subsection{Clear mex functions in workspace}
\lstinline[language=matlab]{clear mex;}

\begin{lstlisting}[language=matlab]
nlhs: number of left hand side;
plhs: pointer of left hand side;
nrhs: number of right hand side;
prhs: pointer of right hand side;
\end{lstlisting}

\subsection{Debug in C++ obselete}
\begin{lstlisting}[language=c++]
1. C/C++->General->Additional Include Directories:
	$(MATLAB_PATH)\MATLAB\R2014b\extern\include
	$(MATLAB_PATH)\MATLAB\R2014b\extern\include\win64
2. Linker->Input:
	libmat.lib
	libmx.lib
	libmex.lib
	libeng.lib
3. Linker->General->Additional Library Directories:
	$(MATLAB_PATH)\MATLAB\R2014b\extern\lib\win64\microsoft
	$(MATLAB_PATH)\MATLAB\R2014b\extern\lib\win32\microsoft
4. Change Debugger to X64
5. Linker->Advanced->Target Machine:
	MachineX64(/MACHINE:X64)
6. add environment to PATH or copy dlls to .exe:
	$(MATLAB_PATH)\MATLAB\R2014b\extern\lib\win64\microsoft
	$(MATLAB_PATH)\MATLAB\R2014b\bin\win64
\end{lstlisting}


\subsection{Debug in C++ within Matlab}
\begin{lstlisting}[language=matlab]
1. Compile target.cpp in '-g' mode 
2. copy tartget.mex target.pdb as well as target.cpp to debug location
3. In VS->Debug->Attach to Process, find Matlab and attach target.cpp file to matlab.
4. run in Matlab and debug.
5. Notice that any change to the target.cpp will cause the debug ineffective.
\end{lstlisting}

\begin{figure}[h]
	\centering
	\includegraphics[width=0.7\linewidth]{attach_to_matlab}
	\caption{}
	\label{fig:attachtomatlab}
\end{figure}

\begin{figure}[h]
\centering
\includegraphics[width=0.7\linewidth]{attach_to_matlab2}
\caption{}
\label{fig:attachtomatlab2}
\end{figure}

\subsection{Useful codes}
\begin{lstlisting}[language=c++]
#include "mex.h" // required

// mex_h macro is defined in mex.h
// MEX_MATLAB_FILE macro is defined when compiling using mex.
plhs[0] = mxCreateDoubleScalar(0.0);
plhs[1] = mxCreateDoubleMatrix(nFaces + nRibs, 1, mxREAL);

int nNodes = (int)mxGetM(prhs[0]);
double const *nodes = mxGetPr(prhs[0]);

spK = mxCreateSparse(n,n,nzmax, mxREAL);
double *sr  = mxGetPr(spK);
mwIndex *ir = mxGetIr(spK);
mwIndex *jc = mxGetJc(spK);

mwSize total_num_of_cells = mxGetNumberOfElements(vvCells);
const mxArray *cell = mxGetCell(vvCells, i);
mwSize n = mxGetN(cell);

double a = mxGetScalar(prls[3]);
\end{lstlisting}

\subsection{matlab mex}
\lstinline[language=matlab]|.m| can also be used to generate mexfunctions( for fast prototype test). There are two ways:
\begin{enumerate}
	\item Use Matlab Coder.
	\item use \lstinline[language=matlab]|codegen| command.
	\begin{lstlisting}[language=matlab]
	p1 = coder.typeof(0,[3,1]);
	p2 = coder.typeof(0, [3,1]);
	p3 = coder.typeof(0,[3,1]);
	E =coder.typeof(0, [1,1]);
	nu = coder.typeof(0, [1,1]);
	eleType = coder.typeof(char(0), [1,Inf]);
	hp = coder.typeof(0, [1,1]);
	
	% Plate3_Compute_keg(p1,p2,p3, g_Ep, g_nup, g_eleType, g_hp);
	codegen '.\stiffenedPlate\src_matlab\Plate3_Compute_keg.m' -config:mex ...
	-args {p1, p2, p3 E, nu, eleType, hp} -d '.\codegen\' -o '.\'
	\end{lstlisting}
\end{enumerate}

To be noticed, the mex functions built by matlab are not efficient enough. Meanwhile, some functions are not supported, such as \lstinline[language=matlab]|tic; toc; sparse; function handle. anonymous functions.|

\section{fmincon}
\subsection{Out of memory problem}
Supply a sparse hessian matrix 
\begin{lstlisting}[language=c++]
'HessPattern', Hstr
% spy(Hstr) % view the sparsity structure of Hstr.
\end{lstlisting}
\subsection{objective function check}
the objective function should converge to lower bounds if there are no constraints presented.

\section{editting colormap}
\begin{figure}[h!]
\centering
\includegraphics[width=0.7\linewidth]{colormapEditor}
\caption{colormap editor}
\label{fig:colormapeditor}
\end{figure}


	
\end{document}