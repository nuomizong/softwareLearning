% !Mode:: "TeX:UTF-8"	% read in as utf8 file.

\chapter{Memory management}
\section{Realloc}
\url{http://www.cplusplus.com/reference/cstdlib/realloc/}
\subsection{function realloc \lstinline[language=c++]|<cstdlib>|}
\begin{lstlisting}[language=c++]
void* realloc (void* ptr, size_t size);
\end{lstlisting}

\subsection{Reallocate memory block}
Changes the size of the memory block pointer to by \textit{ptr}.

The function may move the mermory block to a new location ( whose address is returned by the function).

The content of the memory block is preserved up to the lesser of the new and old sizes, \textcolor{red}{even if the block is moved to a new location}. If the new \textit{size} is larger, the value of the newly allocated portion in indeterminate.

In case that \textit{ptr} is a null pointer, the function behaves like malloc, assigning a new block of \textit{size} bytes and returning a pointer to its beginning.

\subsection{C99/C11(C++11)}
If \textit{size} is zero, the return value depends on the particular library implementation: it may either be a \textit{null} pointer or some other location that shall not be dereferenced.

If the function fails to allocate the requested block of memory, a null pointer is returned, and the memory block pointed to by argument \textit{ptr} is not deallocated (it is still valid, and with its contents unchanged).

\subsection{Parameters}

\begin{description}
\item[ptr] Pointer to a memory block previously allocated with malloc, calloc or realloc. Alternatively, this can be a \textit{null pointer}, in which case a new block is allocated (as if malloc was called).
\item[size] New size for the memory block, in bytes. \lstinline[language=c++]|size_t| is an unsigned integer type.
\end{description}

\subsection{Return Value}
A pointer to the reallocated memory block, which may be either the same as \textit{ptr} or a new location. The type of this pointer is void*, which can be cast to the desired type of data pointer in order to be dereferenceable.

\fbox{A \textit{null-pointer} indicates that the function failed to allocate storage, and thus the block pointed by \textit{ptr} was not modified.}

\subsection{Example}
\begin{lstlisting}[language=c++]
/* realloc example: rememb-o-matic */
#include <cstdio>      /* printf, scanf, puts */
#include <cstdlib>     /* realloc, free, exit, NULL */

int main ()
{
	int input, n;
	int count = 0;
	int * numbers = NULL;
	int * more_numbers = NULL;

	do
	{
		std::printf ("Enter an integer value (0 to end): ");
		std::scanf ("%d", &input);
		count++;

		more_numbers = (int*) std::realloc (numbers, count * sizeof(int));

		if (more_numbers!=NULL)
		{
			numbers=more_numbers;
			numbers[count-1]=input;
		}
		else
		{
			std::free (numbers);
			std::puts ("Error (re)allocating memory");
			exit (1);
		}
	} while (input!=0);

	std::printf ("Numbers entered: ");
	for (n=0;n<count;n++) std::printf ("%d ",numbers[n]);
	std::free (numbers);

	return 0;
}
\end{lstlisting}

The program prompts the user for numbers until a zero character is entered. Each time a new value is introduced the memory block pointed by numbers is increased by the size of an int.

\subsection{Data races}
Only the storage reference by \textit{ptr} and the returned pointer are modified. No other storage locations are accessed by the call.

If the function releases or reuses a unit of storage that is reused or released by another \textit{allocation or deallocation function}, the functions are synchronized in such a way that the deallocation happens entirely before the next allocation.

\subsection{Exceptions(C++)}
\textbf{No-throw guarantee:} this function never throws exceptions.
