% !Mode:: "TeX:UTF-8"	% read in as utf8 file.
\documentclass[10pt,a4paper]{article}

% !Mode:: "TeX:UTF-8"	% read in as utf8 file.
\usepackage{microtype}

% ---- Math packages. ----
\usepackage{amsmath}
\usepackage{amsfonts}
\usepackage{amssymb}
\usepackage{amsthm}	% thm = theorem
% \usepackage{mathtools}
% \usepackage{array}

\usepackage{siunitx} % standard unit
\DeclareSIUnit\rotation{r}

% ---- Figures and Captions. ----
\usepackage{graphicx}


\usepackage{caption}
% captionof command. when inserting graphics without figure environment, captionof can produce the caption.

\usepackage{subcaption} % subfigure environment.

% ---- Tables. ----
\usepackage{booktabs} % three-line tables: toprule, midrule, bottomrule
\usepackage{longtable}
\usepackage{multirow}

% ---- References. ----
\usepackage[square,sort,comma,numbers]{natbib}
%\usepackage{biblatex}

\usepackage{hyperref} % url command

%======================================================
%	Color
%======================================================
\usepackage{color}
\usepackage{colortbl}
\definecolor{bkg}{rgb}{0.95,0.95,0.92}

%======================================================
%	Todo
%======================================================
\usepackage{todonotes}
\newcommand{\TODO}[1]{{\color{red}{[TODO: #1]}}}


\RequirePackage[l2tabu, orthodox]{nag}

%\usepackage{tikz}

\usepackage[width=21.00cm, height=29.70cm, left=2.54cm, right=2.54cm, top=2.54cm, bottom=2.54cm]{geometry}

\usepackage[utf8]{inputenc}

\usepackage{verbatim}
\usepackage{listings} % can print various codes including listings itself and LaTeX. can also use lstinline command.
% ====== set styles for listings. ======
\definecolor{codegreen}{rgb}{0,0.6,0}
\definecolor{codegray}{rgb}{0.5,0.5,0.5}
\definecolor{codepurple}{rgb}{0.58,0,0.82}

\lstdefinestyle{zstyle}{
	backgroundcolor=\color[rgb]{0.95,0.95,0.92},
	commentstyle=\color{codegreen},
	basicstyle=\ttfamily\small,
	keywordstyle=\color{codepurple},
	numberstyle=\tiny\color{codegray},
	numbersep=5pt,
	stringstyle=\color{red},
	showspaces=false,
	showstringspaces=false,
	showtabs=false,
	numbers=left,
	prebreak=\raisebox{0ex}[0ex][0ex]{\ensuremath{\hookleftarrow}},
	captionpos=b,
	frame=single,
	breakatwhitespace=false,
	breaklines=true,
	keepspaces=true,
	tabsize=4,
	%escapeinside,
}

\lstset{style=zstyle}
% ====== set styles for listings. ======

\usepackage{tabularx} % tabularx environment. equivalent lenth?

\usepackage{CJKutf8} % Chinese, Japanese, Korean input with utf-8 encoding. it loads \usepackage[utf8]{inputenc} internally



%\usepackage{fancyvrb} % not familiar with

\usepackage{cleveref}
%% \crefname{ <type> }{ <singular> }{ <plural> }
%% \Cref{ key} capitalize the first letter.
%\crefname{table}{table}{tables} 
%\crefname{figure}{fig.}{figs.}
%\crefname{equation}{eq.}{eqs.}

\usepackage{newverbs} % if listings package doesn't work, use this one to highlight.
\newverbcommand{\cverb}{\color{red}}{} % colored vertb with red
\newverbcommand{\bverb}	% verbatim with gray background
{\begin{lrbox}{\verbbox}}
	{\end{lrbox}\colorbox{gray}{\box\verbbox}}

%======================== separator

\usepackage{txfonts} % piup
\usepackage{xfrac} % more beautiful and standard fractions. \sfrac{1}{2}

%\usepackage{enumerate}

\numberwithin{equation}{section} % number equation with section number.

%\usepackage[numbers,sort]{natbib} % [1,3,2] => [1,2,3]
%\usepackage[numbers,sort & compress]{natbib} % [1,3,2] => [1-3]

%\setlength{\parindent}{0pt}

\usepackage{enumerate} % [i], [ii]...

\begin{document}
\author{Anzong Zheng}
\title{Maya Animation}
\date{May 30, 2016}
\maketitle

\tableofcontents

\newpage

\section{Record a slide show with narration, ink, and slide timings}
Audio narrations and timings can enhance a Web-based or self-running slide show. If you’re planning to create a video with your presentation, using narrations and timings is a great way to make it less static. You can use audio narration to archive a meeting, so that presenters or absentees can review the presentation later and hear any comments made during the presentation.

You can also record your use of the laser pointer in the slide show together with your narrations during a show. To do this see Turn your mouse into a laser pointer.

\subsection{Using narration in a slide show}
You can either record a narration before you run a slide show or record a narration during a slide show and include audience comments in the recording. If you don't want narration throughout the presentation, you can record comments only on selected slides or turn off the narration so that it plays only when you want it to play.

When you add a narration to a slide, a sound icon \includegraphics{ppt_sound} Audio appears on the slide. As with any sound, you can either click the icon to play the sound or set the sound to play automatically.

To record and hear a narration, your computer must be equipped with a sound card, microphone, and speakers.

Before you start recording, PowerPoint 2010 will prompt you to record either just the slide timings, just the narrations, or both at the same time. You can also set the slide timings manually. Slide timings are especially useful if you want the presentation to run automatically with your narration. Recording slide timings will also record the times of animation steps and the use of any triggers on your slide. You can turn the timings off when you don't want the presentation to use them.

\subsection{Record a narration before or during a slide show}
\begin{enumerate}
\item When you record a narration, you run through the presentation and record each slide. You can pause and resume recording any time.
\item Ensure your microphone is set up and in working order prior to recording your slide show.
\item On the \textbf{Slide Show} tab, in the \textbf{Set Up} group, click \textbf{Record Slide Show} \includegraphics{ppt_record_slide_show}
\item Select one the following:
\begin{itemize}
\item Start Recording from Beginning
\item Start Recording from Current Slide
\end{itemize}
\item In the \textbf{Record  Slide Show} diaglog box, select the \textbf{Narrations and laser pointer} check box, and if appropriate, select or deselect the \textbf{Slide and animation timings} check box.
\item \textbf{Click Start Recording}.
Tip: To pause the narration, in the \textbf{Recording} shortcut menu, click \textbf{Pause}. And to resume your narration, click \textbf{Resume Recording}.
\item To end your slide show recording, right click the slide, and then click \textbf{End Show}.
\item The recorded slide show timings are automatically saved and the slide show appears in Slide Sorter view with timings beneath each slide.
\end{enumerate}

\section{Save presentation as a video}
\begin{figure}[h]
\centering
\includegraphics[scale=0.5]{ppt_presentation_2_video}
\caption{save presentation as a video}
\label{fig:ppt_presentation_2_video}
\end{figure}

\begin{enumerate}
\item Create presentation.
\item (\textbf{Optional}) \textbf{Record and add narration and timings to a slide show} and \textbf{Turn mouse into a laser pointer}.
\item Save the presentation.
\item On the \textbf{File} menu, click \textbf{Click \& Send}.
\item Under \textbf{Save \& Send}, click \textbf{Create a video}.
\item To display all video quality and size options, under \textbf{Create a video}, click the \textbf{Computer \& HD Displays}
\item Do one of the following:
\begin{itemize}
\item To create a video with very high quality, yet a large file size, click \textbf{Computer \& HD Displays}.
\item To create a video with moderate file size and medium quality, click \textbf{Internet \& DVD}.
\item To create a video with smallest file size, ye low quality, click \textbf{Portable Devices}.
\end{itemize}
\item Click the \textbf{Don't Use Recorded Timings and Narrations} down arrow and then, do one of the following:
\begin{itemize}
\item If you did not record and time voice narration and laser pointer movements, click \textbf{Don't Use Recorded Timings and Narration}.
\item If you recorded and timed narration and pointer movements, click \textbf{Use Recorded Timings and Narrations}.
\end{itemize}
\item Click \textbf{Create Video}
\item In the \textbf{File name} box, enter a file name for the video, browse for folder that will contain this file, and then click Save. You can track the progress of the video creation by looking at the status bar at the bottom of your screen. The video creation process can take up to several hours depending on the length of the video and the complexity of the presentation.
\end{enumerate}

\end{document}