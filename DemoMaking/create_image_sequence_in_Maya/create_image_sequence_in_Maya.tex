% !Mode:: "TeX:UTF-8"	% read in as utf8 file.
\documentclass[10pt,a4paper]{article}

% !Mode:: "TeX:UTF-8"	% read in as utf8 file.
\usepackage{microtype}

% ---- Math packages. ----
\usepackage{amsmath}
\usepackage{amsfonts}
\usepackage{amssymb}
\usepackage{amsthm}	% thm = theorem
% \usepackage{mathtools}
% \usepackage{array}

\usepackage{siunitx} % standard unit
\DeclareSIUnit\rotation{r}

% ---- Figures and Captions. ----
\usepackage{graphicx}


\usepackage{caption}
% captionof command. when inserting graphics without figure environment, captionof can produce the caption.

\usepackage{subcaption} % subfigure environment.

% ---- Tables. ----
\usepackage{booktabs} % three-line tables: toprule, midrule, bottomrule
\usepackage{longtable}
\usepackage{multirow}

% ---- References. ----
\usepackage[square,sort,comma,numbers]{natbib}
%\usepackage{biblatex}

\usepackage{hyperref} % url command

%======================================================
%	Color
%======================================================
\usepackage{color}
\usepackage{colortbl}
\definecolor{bkg}{rgb}{0.95,0.95,0.92}

%======================================================
%	Todo
%======================================================
\usepackage{todonotes}
\newcommand{\TODO}[1]{{\color{red}{[TODO: #1]}}}


\RequirePackage[l2tabu, orthodox]{nag}

%\usepackage{tikz}

\usepackage[width=21.00cm, height=29.70cm, left=2.54cm, right=2.54cm, top=2.54cm, bottom=2.54cm]{geometry}

\usepackage[utf8]{inputenc}

\usepackage{verbatim}
\usepackage{listings} % can print various codes including listings itself and LaTeX. can also use lstinline command.
% ====== set styles for listings. ======
\definecolor{codegreen}{rgb}{0,0.6,0}
\definecolor{codegray}{rgb}{0.5,0.5,0.5}
\definecolor{codepurple}{rgb}{0.58,0,0.82}

\lstdefinestyle{zstyle}{
	backgroundcolor=\color[rgb]{0.95,0.95,0.92},
	commentstyle=\color{codegreen},
	basicstyle=\ttfamily\small,
	keywordstyle=\color{codepurple},
	numberstyle=\tiny\color{codegray},
	numbersep=5pt,
	stringstyle=\color{red},
	showspaces=false,
	showstringspaces=false,
	showtabs=false,
	numbers=left,
	prebreak=\raisebox{0ex}[0ex][0ex]{\ensuremath{\hookleftarrow}},
	captionpos=b,
	frame=single,
	breakatwhitespace=false,
	breaklines=true,
	keepspaces=true,
	tabsize=4,
	%escapeinside,
}

\lstset{style=zstyle}
% ====== set styles for listings. ======

\usepackage{tabularx} % tabularx environment. equivalent lenth?

\usepackage{CJKutf8} % Chinese, Japanese, Korean input with utf-8 encoding. it loads \usepackage[utf8]{inputenc} internally



%\usepackage{fancyvrb} % not familiar with

\usepackage{cleveref}
%% \crefname{ <type> }{ <singular> }{ <plural> }
%% \Cref{ key} capitalize the first letter.
%\crefname{table}{table}{tables} 
%\crefname{figure}{fig.}{figs.}
%\crefname{equation}{eq.}{eqs.}

\usepackage{newverbs} % if listings package doesn't work, use this one to highlight.
\newverbcommand{\cverb}{\color{red}}{} % colored vertb with red
\newverbcommand{\bverb}	% verbatim with gray background
{\begin{lrbox}{\verbbox}}
	{\end{lrbox}\colorbox{gray}{\box\verbbox}}

%======================== separator

\usepackage{txfonts} % piup
\usepackage{xfrac} % more beautiful and standard fractions. \sfrac{1}{2}

%\usepackage{enumerate}

\numberwithin{equation}{section} % number equation with section number.

%\usepackage[numbers,sort]{natbib} % [1,3,2] => [1,2,3]
%\usepackage[numbers,sort & compress]{natbib} % [1,3,2] => [1-3]

%\setlength{\parindent}{0pt}

\usepackage{enumerate} % [i], [ii]...

\begin{document}
\author{Anzong Zheng}
\title{Maya Animation}
\date{May 30, 2016}
\maketitle

\tableofcontents

\newpage

\section{Create Image Sequences in Maya}
\subsection{Set Playback option}
Change the \textbf{Playback option} like this:

\begin{figure}[h]
\centering
\includegraphics[width=0.7\linewidth]{playbackOption}
\caption{set playback option}
\label{fig:playbackOption}
\end{figure}

\subsection{Set, Mute, Delete specific Keys}
\subsubsection{Set Key}
Select \textbf{Camera} or \textbf{Object}, then goto \textbf{Attribute Editor}, \textbf{Right Click} at interested property and choose \textbf{Key Selected}. If selected, the property's background will turn to \textcolor{red}{Red}. Originally, they have \textcolor{black}{black} background.

I choose \textbf{Rotation Y} as the selected key and set it as 0 at the keyframe 1, as shown in \ref{fig:selectedKey_frame1}

\begin{figure}[h]
\centering
\includegraphics[scale=0.5]{selectedKey_frame1}
\caption{selected key frame 1}
\label{fig:selectedKey_frame1}
\end{figure}

I also set the \textbf{Rotation Y} key at the final keyframe as \ref{fig:selectedKey_framen}:
\begin{figure}[h]
\centering
\includegraphics[scale=0.5]{selectedKey_framen}
\caption{selected key frame n}
\label{fig:selectedKey_framen}
\end{figure}

The interpreted \textbf{Rotation Y} key can be interpreted as \ref{fig:selectedKey_framex}:
%\begin{figure}[h]
%\centering
\begin{center}
\includegraphics[scale=0.5]{selectedKey_framex}
\captionof{figure}{selected key frame}\label{fig:selectedKey_framex}
\end{center}

%\end{figure}

\subsubsection{Mute Key}
Muting lets you isolate and focus on a specific motion. For example, working with a model whose arms and legs are animated, you can use muting to turn off the animation of either the arms or legs.

You can mute selected animation channels to temporarily disable the animation without disconnecting its curve from the animated object. See Mute channels. You can also mute keys for a selected object. See Mute keys.

\begin{figure}[h]
\centering
\includegraphics[scale=0.3]{set_mute_del_keyframes}
\caption{Set, Mute and Delete Key}
\label{fig:set_mute_del_keyframes}
\end{figure}

\subsubsection{Select all Keys}
After object or camera selected, press \textbf{s: Animate $ \rightarrow $ Set Key}, then all properties are selected.

\subsection{file output option}
To make an animation or do batch rendering, I will need to export multiple images, I choose to export \textbf{Jpeg} image format and set the output file name as the following \ref{fig:file_output_option}
\begin{figure}[h]
\centering
\includegraphics[scale=0.5]{file_output_option}
\caption{file output option}
\label{fig:file_output_option}
\end{figure}

\subsection{Set Project location}
Go to \textbf{File} and \textbf{Set Project...}, choose a folder and make it where the project saves.

\subsection{Set frame range}
Set the \textbf{Frame range} the same as the \textbf{Playback option}.

\begin{figure}[h]
\centering
\includegraphics[scale=0.5]{frameRange}
\caption{frame range}
\label{fig:frameRange}
\end{figure}

\subsection{Set Renderable camera}
Set the \textbf{Renderable camera} the same as the camera I am using.
%\begin{figure}[h]
%\centering
\begin{center}
\includegraphics[scale=0.4]{renderableCamera}
\captionof{figure}{set renderable camera}
\label{fig:renderableCamera}
\end{center}

%\end{figure}

\subsection{Output images}
\begin{figure}[h]
\centering
\includegraphics[scale=0.5]{outputImages}
\caption{output images}
\label{fig:outputImages}
\end{figure}


\end{document}