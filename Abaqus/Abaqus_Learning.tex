% !Mode:: "TeX:UTF-8"	% read in as utf8 file.
\documentclass[10pt,a4paper]{article}

% !Mode:: "TeX:UTF-8"	% read in as utf8 file.
\usepackage{microtype}

% ---- Math packages. ----
\usepackage{amsmath}
\usepackage{amsfonts}
\usepackage{amssymb}
\usepackage{amsthm}	% thm = theorem
% \usepackage{mathtools}
% \usepackage{array}

\usepackage{siunitx} % standard unit
\DeclareSIUnit\rotation{r}

% ---- Figures and Captions. ----
\usepackage{graphicx}


\usepackage{caption}
% captionof command. when inserting graphics without figure environment, captionof can produce the caption.

\usepackage{subcaption} % subfigure environment.

% ---- Tables. ----
\usepackage{booktabs} % three-line tables: toprule, midrule, bottomrule
\usepackage{longtable}
\usepackage{multirow}

% ---- References. ----
\usepackage[square,sort,comma,numbers]{natbib}
%\usepackage{biblatex}

\usepackage{hyperref} % url command

%======================================================
%	Color
%======================================================
\usepackage{color}
\usepackage{colortbl}
\definecolor{bkg}{rgb}{0.95,0.95,0.92}

%======================================================
%	Todo
%======================================================
\usepackage{todonotes}
\newcommand{\TODO}[1]{{\color{red}{[TODO: #1]}}}


\RequirePackage[l2tabu, orthodox]{nag}

%\usepackage{tikz}

\usepackage[width=21.00cm, height=29.70cm, left=2.54cm, right=2.54cm, top=2.54cm, bottom=2.54cm]{geometry}

\usepackage[utf8]{inputenc}

\usepackage{verbatim}
\usepackage{listings} % can print various codes including listings itself and LaTeX. can also use lstinline command.
% ====== set styles for listings. ======
\definecolor{codegreen}{rgb}{0,0.6,0}
\definecolor{codegray}{rgb}{0.5,0.5,0.5}
\definecolor{codepurple}{rgb}{0.58,0,0.82}

\lstdefinestyle{zstyle}{
	backgroundcolor=\color[rgb]{0.95,0.95,0.92},
	commentstyle=\color{codegreen},
	basicstyle=\ttfamily\small,
	keywordstyle=\color{codepurple},
	numberstyle=\tiny\color{codegray},
	numbersep=5pt,
	stringstyle=\color{red},
	showspaces=false,
	showstringspaces=false,
	showtabs=false,
	numbers=left,
	prebreak=\raisebox{0ex}[0ex][0ex]{\ensuremath{\hookleftarrow}},
	captionpos=b,
	frame=single,
	breakatwhitespace=false,
	breaklines=true,
	keepspaces=true,
	tabsize=4,
	%escapeinside,
}

\lstset{style=zstyle}
% ====== set styles for listings. ======

\usepackage{tabularx} % tabularx environment. equivalent lenth?

\usepackage{CJKutf8} % Chinese, Japanese, Korean input with utf-8 encoding. it loads \usepackage[utf8]{inputenc} internally



%\usepackage{fancyvrb} % not familiar with

\usepackage{cleveref}
%% \crefname{ <type> }{ <singular> }{ <plural> }
%% \Cref{ key} capitalize the first letter.
%\crefname{table}{table}{tables} 
%\crefname{figure}{fig.}{figs.}
%\crefname{equation}{eq.}{eqs.}

\usepackage{newverbs} % if listings package doesn't work, use this one to highlight.
\newverbcommand{\cverb}{\color{red}}{} % colored vertb with red
\newverbcommand{\bverb}	% verbatim with gray background
{\begin{lrbox}{\verbbox}}
	{\end{lrbox}\colorbox{gray}{\box\verbbox}}

%======================== separator

\usepackage{txfonts} % piup
\usepackage{xfrac} % more beautiful and standard fractions. \sfrac{1}{2}

%\usepackage{enumerate}

\numberwithin{equation}{section} % number equation with section number.

%\usepackage[numbers,sort]{natbib} % [1,3,2] => [1,2,3]
%\usepackage[numbers,sort & compress]{natbib} % [1,3,2] => [1-3]

%\setlength{\parindent}{0pt}

\usepackage{enumerate} % [i], [ii]...
%% !Mode:: "TeX:UTF-8"	% read in as utf8 file.
\usepackage{microtype}

% ---- Math packages. ----
\usepackage{amsmath}
\usepackage{amsfonts}
\usepackage{amssymb}
\usepackage{amsthm}	% thm = theorem
% \usepackage{mathtools}
% \usepackage{array}

\usepackage{siunitx} % standard unit
\DeclareSIUnit\rotation{r}

% ---- Figures and Captions. ----
\usepackage{graphicx}


\usepackage{caption}
% captionof command. when inserting graphics without figure environment, captionof can produce the caption.

\usepackage{subcaption} % subfigure environment.

% ---- Tables. ----
\usepackage{booktabs} % three-line tables: toprule, midrule, bottomrule
\usepackage{longtable}
\usepackage{multirow}

% ---- References. ----
\usepackage[square,sort,comma,numbers]{natbib}
%\usepackage{biblatex}

\usepackage{hyperref} % url command

%======================================================
%	Color
%======================================================
\usepackage{color}
\usepackage{colortbl}
\definecolor{bkg}{rgb}{0.95,0.95,0.92}

%======================================================
%	Todo
%======================================================
\usepackage{todonotes}
\newcommand{\TODO}[1]{{\color{red}{[TODO: #1]}}}


\RequirePackage[l2tabu, orthodox]{nag}

%\usepackage{tikz}

\usepackage[width=21.00cm, height=29.70cm, left=2.54cm, right=2.54cm, top=2.54cm, bottom=2.54cm]{geometry}

\usepackage[utf8]{inputenc}

\usepackage{verbatim}
\usepackage{listings} % can print various codes including listings itself and LaTeX. can also use lstinline command.
% ====== set styles for listings. ======
\definecolor{codegreen}{rgb}{0,0.6,0}
\definecolor{codegray}{rgb}{0.5,0.5,0.5}
\definecolor{codepurple}{rgb}{0.58,0,0.82}

\lstdefinestyle{zstyle}{
	backgroundcolor=\color[rgb]{0.95,0.95,0.92},
	commentstyle=\color{codegreen},
	basicstyle=\ttfamily\small,
	keywordstyle=\color{codepurple},
	numberstyle=\tiny\color{codegray},
	numbersep=5pt,
	stringstyle=\color{red},
	showspaces=false,
	showstringspaces=false,
	showtabs=false,
	numbers=left,
	prebreak=\raisebox{0ex}[0ex][0ex]{\ensuremath{\hookleftarrow}},
	captionpos=b,
	frame=single,
	breakatwhitespace=false,
	breaklines=true,
	keepspaces=true,
	tabsize=4,
	%escapeinside,
}

\lstset{style=zstyle}
% ====== set styles for listings. ======

\usepackage{tabularx} % tabularx environment. equivalent lenth?

\usepackage{CJKutf8} % Chinese, Japanese, Korean input with utf-8 encoding. it loads \usepackage[utf8]{inputenc} internally



%\usepackage{fancyvrb} % not familiar with

\usepackage{cleveref}
%% \crefname{ <type> }{ <singular> }{ <plural> }
%% \Cref{ key} capitalize the first letter.
%\crefname{table}{table}{tables} 
%\crefname{figure}{fig.}{figs.}
%\crefname{equation}{eq.}{eqs.}

\usepackage{newverbs} % if listings package doesn't work, use this one to highlight.
\newverbcommand{\cverb}{\color{red}}{} % colored vertb with red
\newverbcommand{\bverb}	% verbatim with gray background
{\begin{lrbox}{\verbbox}}
	{\end{lrbox}\colorbox{gray}{\box\verbbox}}

%======================== separator

\usepackage{txfonts} % piup
\usepackage{xfrac} % more beautiful and standard fractions. \sfrac{1}{2}

%\usepackage{enumerate}

\numberwithin{equation}{section} % number equation with section number.

%\usepackage[numbers,sort]{natbib} % [1,3,2] => [1,2,3]
%\usepackage[numbers,sort & compress]{natbib} % [1,3,2] => [1-3]

%\setlength{\parindent}{0pt}

\usepackage{enumerate} % [i], [ii]...

\begin{document}
\author{Anzong Zheng}
\title{Learning Abaqus}
\date{January 20, 2015}
\maketitle

\tableofcontents

\newpage

\section{Operations}
\begin{itemize}
	\item Move object: \textbf{Ctrl + alt + middle button of Mouse}
	\item Rotate object: \textbf{Ctrl + alt + left button of Mouse}
\end{itemize}

\section{A simple introduction to Python}
\subsection{Start the Python interpreter}
Python is an interpreted language. This means you can type a statement and view the results without having to compile and link your scripts.

If youhave Abaqus installed on your UNIX or Windows workstation, type \textbf{abaqus python} at the system prompt. Python enters its interpretive mode and displays the \textgreater \textgreater \textgreater prompt.

\subsection{See the value of a variable or expression}
You can enter Python statements at the \textgreater \textgreater \textgreater prompt. To see the value of a variable or expression, type the variable name or expression at the Python prompt. 

\begin{figure}[h]
\centering
\includegraphics[width=0.7\linewidth]{"Start Python interpretive mode"}
\caption{Start Python interpretive mode} \label{fig:StartPythoninterpretivemode}
\vspace{\baselineskip}
\end{figure}

\subsection{Exit Python interpreter}
To exit the Python Interpreter, type \textbf{[Ctrl+D]} on UNIX systems or \textbf{[Ctrl]+Z[Enter]} on Windows systems.

\section{Run Abaqus Python for results}
\subsection{Run Python script}
You can also use Python to run a script directly by typing \textbf{abaqus python} \textit{scriptname}\textbf{.py} at the system prompt. Abaqus will run the script through the Python interpreter and return you to the system prompt.

\section{Import job via .inp file}
In \textbf{Model module}, double click \textbf{Jobds} under \textbf{Analysis} to create a new job. In the \textbf{Drop-down arrow of Source}, choose \textbf{Input File} then select the targeted .inp file.

\textbf{NOTE:} this method not only can import the whole model, but also can specify the job.

\section{Running a script when you start Abaqus/CAE}
You can run a script when you start an Abaqus/CAE session by typing the following command:

\noindent\vspace{1em}\hrule
\begin{verbatim}
abaqus cae script=myscript.py
\end{verbatim}
\noindent\hrule\vspace{1em}

where \textbf{myscript.py} is the name of the file containing the script. The equivalent command for Abaqus/Viewer is

\begin{lstlisting}[language=command.com]
abaqus viewer script=myscript.py
\end{lstlisting}

Arguments can be passed into the script by entering -- on the command line, followed by the arguments separated by one or more spaces. These arguments will be ignored by the Abaqus/CAE execution procedure, but they will be accessible within the script. For more information, see "Abaqus/CAE execution," Section 3.2.5 of the Abaqus Analysis User's Manual, and "Abaqus/Viewer execution," Section 3.2.6 of the Abaqus Analysis User's Manual.

\section{Export stiffness matrix}
By adding following lines in the inp file  just after \lstinline|** STEP: staticAnalysis|
\begin{lstlisting}
*STEP, name=exportmatrix
*MATRIX GENERATE, STIFFNESS
*MATRIX OUTPUT, STUFFNESS, FORMAT=COORDINATE
*END STEP
\end{lstlisting}
\section{Abaqus/CAE execution}
\subsection{Overview}
Abaqus/CAE, an interactive environment for creating, submitting, monitoring, and evaluating results from Abaqus simulations, is executed by runing the Abaqus execution procedure and specifying the \textbf{cae} parameter.

\subsection{Command summary}
\begin{lstlisting}[language=command.com]
abaqus cae       [database=databse-file]
                 [replay=replay-file]
                 [recover=journal-file]
                 [startup=startup-file]
                 [script=script-file]
                 [noGUI=[noGUI-file]]
                 [noenvstartup]
                 [noSavedOptions]
                 [noSavedGuiOptions]
                 [noStartupDialog]
                 [custom=script-file]
                 [guiTester[GUI-script]]
                 [guiRecord]
                 [guiNoRecord]
\end{lstlisting}

\section{beam}
\begin{figure}[h!]
\centering
\includegraphics[width=0.7\linewidth]{beam_crosssection}
\caption{beam cross-setion}
\label{fig:beamcrosssection}
\end{figure}

\begin{figure}[h!]
\centering
\includegraphics[width=0.7\linewidth]{export_beam}
\caption{export beam obj}
\label{fig:exportbeam}
\end{figure}

\begin{figure}[h!]
\centering
\includegraphics[width=0.7\linewidth]{view_beam_1}
\caption{view beam 1}
\label{fig:viewbeam1}
\end{figure}

\begin{figure}[h!]
\centering
\includegraphics[width=0.7\linewidth]{view_beam_2}
\caption{view beam 2}
\label{fig:viewbeam2}
\end{figure}

\section{View}
\begin{figure}[h!]
	\centering
	\includegraphics[width=0.5\linewidth]{view_tools2}
	\caption{view tools 2}
	\label{fig:viewstool}
\end{figure}

\begin{figure}[h!]
\centering
\includegraphics[width=0.5\linewidth]{views_tool}
\caption{view tools}
\label{fig:viewstool}
\end{figure}

\begin{figure}[h!]
	\centering
	\includegraphics[width=0.5\linewidth]{view_continuous}
	\caption{view continuous}
	\label{fig:viewstool}
\end{figure}

Use \textbf{Ctrl} + print and choose corresponding format, like png or others.

\begin{figure}[h!]
\centering
\includegraphics[width=0.5\linewidth]{Savefigure}
\caption{save figure.}
\label{fig:savefigure}
\end{figure}

No edges:

\begin{figure}[h!]
\centering
\includegraphics[width=0.5\linewidth]{no_edges}
\caption{no edges}
\label{fig:noedges}
\end{figure}


\end{document}