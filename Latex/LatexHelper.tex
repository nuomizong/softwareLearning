% !Mode:: "TeX:UTF-8"	% read in as utf8 file.

\usepackage{microtype}
\usepackage[utf8]{inputenc}
%\usepackage{CJKutf8} % Chinese, Japanese, Korean input with utf-8 encoding. it loads \usepackage[utf8]{inputenc} internally
%\usepackage[width=21.00cm, height=29.70cm, left=2.54cm, right=2.54cm, top=2.54cm, bottom=2.54cm]{geometry}

%======================================================
%	Platform packages
%
%\ifwindows
% add settings
%\fi
%\iflinux
% add settings
%\fi
%\ifmacosx
% add settings
%\if
%======================================================
%\usepackage{pdftexcmds}
%\usepackage{catchfile}
%\usepackage{ifluatex}
%\usepackage{ifplatform}

%======================================================
%	Math packages
%======================================================
\usepackage{amssymb}
\usepackage{amsmath}
\usepackage{amsfonts}
\usepackage{amsthm}	% thm = theorem
\usepackage{enumerate} % [i], [ii]...
% \usepackage{mathtools}
% \usepackage{array}
% \usepackage{alltt}
% \usepackage{gensymb}

%======================================================
%	standard unit
%======================================================
\usepackage{siunitx}
\DeclareSIUnit\rotation{r}

%======================================================
%	Pseudo code
%======================================================
% must follow natbib, don't know why.
\usepackage[ruled]{algorithm2e} % add three rules to pseudo code, also used for algorithms

%\renewcommand{\algorithmicrequire} {\textbf{Input:}}
%\renewcommand{\algorithmicensure}{\textbf{Output:}}

%======================================================
%	Equations
%======================================================
\numberwithin{equation}{section} % number equation with section number.

%======================================================
%	Figures and Captions
%======================================================
\usepackage{graphicx}

\usepackage{caption}
% captionof command. when inserting graphics without figure environment, captionof can produce the caption.

\usepackage{subcaption} % subfigure environment.

%======================================================
%	Tables
%======================================================
\usepackage{booktabs} % three-line tables: toprule, midrule, bottomrule
\usepackage{longtable}
\usepackage{multirow}

\usepackage{tabularx} % tabularx environment. equivalent lenth?

%======================================================
%	References
%======================================================
%\usepackage[square,sort,comma,numbers]{natbib}
\usepackage{natbib}
% \bibliographystyle{agsm} harvard type.

%\usepackage[numbers,sort]{natbib} % [1,3,2] => [1,2,3]
%\usepackage[numbers,sort & compress]{natbib} % [1,3,2] => [1-3]
%\usepackage{biblatex}

%======================================================
%	Link
%======================================================
\usepackage{hyperref} % url command

%======================================================
%	Color
%======================================================
\usepackage{color}
\usepackage{colortbl}
\definecolor{bkg}{rgb}{0.95,0.95,0.92}

%======================================================
%	Todo
%======================================================
\usepackage{todonotes}
% Hide or show comments
\newif\ifshowcomments
\showcommentstrue
%\showcommentsfalse
\ifshowcomments
\newcommand{\TODO}[1]{{\color{red}{[TODO: #1]}}}
\else
\newcommand{\TODO}[1]{}
\fi

%======================================================
%	Corrections.
%======================================================
%\usepackage{ulem} % support for strikethrough \sout

% show or hide deleted words or sentences
\newif\ifShowDeleted
\ShowDeletedtrue
%\ShowDeletedfalse
\ifShowDeleted
\newcommand{\delete}[1]{{\color[rgb]{0.2,0.8,0.2}{\sout{#1}}}}
\else
\newcommand{\delete}[1]{}
\fi

% highlight or not added words or sentences
\newif\ifHighlightAdded
\HighlightAddedtrue
%\HighlightAddedfalse
\ifHighlightAdded
\newcommand{\add}[1]{{\color{red}{#1}}}
\else
\newcommand{\add}[1]{#1}
\fi

% replace words by another words
\newif\ifHighlightReplaced
\HighlightReplacedtrue
%\HighlightReplacedfalse
\ifHighlightReplaced
\newcommand{\replace}[2]{{\color{red}{#1}}{\color{green}{ (\sout{#2})}}}
\else
\newcommand{\replace}[2]{#1}
\fi

\RequirePackage[l2tabu, orthodox]{nag}

\usepackage{verbatim}

%======================================================
%	track of changes.
%======================================================
\usepackage{changes}

%======================================================
%	Code listing.
%======================================================
\usepackage{listings} % can print various codes including listings itself and LaTeX. can also use lstinline command.
% ====== set styles for listings. ======
\definecolor{codegreen}{rgb}{0,0.6,0}
\definecolor{codegray}{rgb}{0.5,0.5,0.5}
\definecolor{codepurple}{rgb}{0.58,0,0.82}

\lstdefinestyle{zstyle}{
	backgroundcolor=\color[rgb]{0.95,0.95,0.92},
	commentstyle=\color{codegreen},
	basicstyle=\ttfamily\small,
	keywordstyle=\color{codepurple},
	numberstyle=\tiny\color{codegray},
	numbersep=5pt,
	stringstyle=\color{red},
	showspaces=false,
	showstringspaces=false,
	showtabs=false,
	numbers=left,
	prebreak=\raisebox{0ex}[0ex][0ex]{\ensuremath{\hookleftarrow}},
	captionpos=b,
	frame=single,
	breakatwhitespace=false,
	breaklines=true,
	keepspaces=true,
	tabsize=4,
	%escapeinside,
}

\lstset{style=zstyle}
% ====== set styles for listings. ======

\usepackage{newverbs} % if listings package doesn't work, use this one to highlight.
\newverbcommand{\cverb}{\color{red}}{} % colored vertb with red
\newverbcommand{\bverb}	% verbatim with gray background
{\begin{lrbox}{\verbbox}}
	{\end{lrbox}\colorbox{gray}{\box\verbbox}}