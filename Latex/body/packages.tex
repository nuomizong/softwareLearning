% !Mode:: "TeX:UTF-8"

\chapter{Packages}

\section{useful packages}
\begin{lstlisting}[language={[LaTex]Tex}]
% !Mode:: "TeX:UTF-8"	% read in as utf8 file.

\usepackage{microtype}

\usepackage{amsmath}
\usepackage{amsfonts}
\usepackage{amssymb}
\usepackage{amsthm}	% thm = theorem

\usepackage{hyperref} % url command

\usepackage{todonotes}

\usepackage{biblatex}

\RequirePackage[l2tabu, orthodox]{nag}

\usepackage{tikz}

\usepackage[width=21.00cm, height=29.70cm, left=2.54cm, right=2.54cm, top=2.54cm, bottom=2.54cm]{geometry}

\usepackage[utf8]{inputenc}

\usepackage{booktabs} % three-line tables: toprule, midrule, bottomrule
\usepackage{array}

\usepackage{siunitx}
\DeclareSIUnit\rotation{r}

\usepackage{mathtools}

\usepackage{graphicx}
\graphicspath{{figures/}}

\usepackage{color}
\usepackage{colortbl}
\definecolor{bkg}{rgb}{0.95,0.95,0.92}

\usepackage{listings} % can print various codes including listings itself and LaTeX. can also use lstinline command.
% ====== set styles for listings. ======
\definecolor{codegreen}{rgb}{0,0.6,0}
\definecolor{codegray}{rgb}{0.5,0.5,0.5}
\definecolor{codepurple}{rgb}{0.58,0,0.82}

\lstdefinestyle{zstyle}{
backgroundcolor=\color[rgb]{0.95,0.95,0.92},
commentstyle=\color{codegreen},
basicstyle=\ttfamily\small,
keywordstyle=\color{codepurple},
numberstyle=\tiny\color{codegray},
numbersep=5pt,
stringstyle=\color{red},
showspaces=false,
showstringspaces=false,
showtabs=false,
numbers=left,
prebreak=\raisebox{0ex}[0ex][0ex]{\ensuremath{\hookleftarrow}},
captionpos=b,
frame=single,
breakatwhitespace=false,
breaklines=true,
keepspaces=true,
tabsize=4,
%escapeinside,
}

\lstset{style=zstyle}
% ====== set styles for listings. ======

\usepackage{tabularx} % tabularx environment. equivalent lenth?

\usepackage{CJKutf8} % Chinese, Japanese, Korean input with utf-8 encoding. it loads \usepackage[utf8]{inputenc} internally



\usepackage{longtable}

\usepackage{subcaption} % subfigure environment.

\usepackage{caption}  % captionof command. when inserting graphics without figure environment, captionof can produce the caption.

\usepackage{multirow}

%\usepackage{fancyvrb} % not familiar with

\usepackage{cleveref}
% \crefname{ <type> }{ <singular> }{ <plural> }
% \Cref{ key} capitalize the first letter.
\crefname{table}{table}{tables} 
\crefname{figure}{fig.}{figs.}
\crefname{equation}{eq.}{eqs.}

\usepackage{newverbs} % if listings package doesn't work, use this one to highlight.
\newverbcommand{\cverb}{\color{red}}{} % colored vertb with red
\newverbcommand{\bverb}	% verbatim with gray background
{\begin{lrbox}{\verbbox}}
{\end{lrbox}\colorbox{gray}{\box\verbbox}}

%======================== separator

\usepackage{txfonts} % piup
\usepackage{xfrac} % more beautiful and standard fractions. \sfrac{1}{2}

%\usepackage{enumerate}

\numberwithin{equation}{section} % number equation with section number.

%\usepackage[numbers,sort]{natbib} % [1,3,2] => [1,2,3]
\usepackage[numbers,sort&compress]{natbib} % [1,3,2] => [1-3]


\end{lstlisting}

\section{microtype}
\textbf{microtype} plays with ever-so-slightly shrinking and stretching of the fonts and with the extent to which text protrudes into the margins in a way that yields results that look better, that have fewer instances of hyphenation, and fewer overfull hboxes. It doesn't work with \textbf{latex}, you have to use \textbf{pdflatex} instead. It also works with \textbf{lualatex} and (protrusion only) with \textbf{xelatex}.

\begin{lstlisting}[language={[LaTeX]TeX}]
\usepackage[stretch=10]{microtype} % this allows font expansion up to 1% (default is 2%)
\end{lstlisting}

\section{AMS math packages}
The family of \textbf{AMS math packages}. At least \textbf{amsmath} and \textbf{amssymb}. Also \textbf{amsthm} if I need theorems and the class I'm using doesn't already define them.

Particularly for writing equations, the AMS packages define a rich set of environments to group and align formulas in many different and useful ways. I also like that it encourages the use of semantic commands (e.g. the \textbf{cases} environment) over syntactic commands (e.g. a \lstinline[language={[LaTeX]TeX}]|\left{| followed by an array).

In particular, \begin{itemize}
	\item \textbf{amsthm} provides an easy way to set up different theorem styles
	\item \textbf{amsmath} provides the \lstinline[language={[LaTeX]TeX}]|text| command
	\item \textbf{amssymb} contains several often-used symbols.
\end{itemize}

Its documentation can be found running \textbf{texdoc amsldoc} on a command line.

\section{hyperref}
\textbf{hyperref} is used for setting PDF metadata and to create links, both within the document and for clickable URLs. Even Elsevier has used \textbf{urlbst} to update their bibliography style to support URLs and DOIs; \textbf{hyperref} does the actual work of rendering \textbf{url=} and \textbf{doi=} BibTeX into clickable PDF links.

\section{todonotes}
The \textbf{todonotes} package is a must have in all documents.

\begin{lstlisting}[language={[LaTeX]TeX}]
\usepackage{todonotes}
\end{lstlisting}

The package enables you to insert small notes in the text marking things to do in the document. Something like

\begin{lstlisting}[language={[LaTeX]TeX}]
\todo{Rewrite this answer \ldots}
\end{lstlisting}

At any location in the document a list of the inserted notes can be generated with the 

\begin{lstlisting}[language={[LaTeX]TeX}]
\listoftodos
\end{lstlisting}

command.

\section{biblatex}
For citations and biblographies, \textbf{biblatex} is the package of my choice. Key points:

\begin{itemize}
	\item \textbf{biblatex} includes a wide variety of built-in citation/bibliography styles (numeric, alphabetic, author-year, author-title, verbose [full in-text-citations], with numerous variants for each one). A number of custom styles have been published.
	\item Modifications of the built-in or custom styles can be accomplished using \LaTeX{} macros instead of having to resort to the BibTex programming language.
	\item \textbf{biblatex} offers well-nigh every feature of other bibliography-related \LaTeX{} packages (e.g. multiple/subdivided bibliographies, sorted/compressed citations, entry sets, ibidem functionality, back references). If a feature is not included, chances are high it is on the package authors' to-do list.
	
	\item The \textbf{babel} package is supported, and \textbf{biblatex} comes with localization files for about a dozen languages (with the list still growing).
	
	\item Although the current version of \textbf{biblatex} (2.8a) still allows to use BibTeX as a database backend, by default it cooperates with Biber which supports bibliographies using Unicode. Biber (currently at version 1.8) is included in TeX Live and MiKTeX. Many features introduced since \textbf{biblatex} 1.1 (e.g., advanced name disambiguation, smart crossref data inheritance, configurable sorting schemes, dynamic datasource modification) are "Biber only".
\end{itemize}

\section{nag}
One package that’s \textit{really} general purpose is \textbf{nag}: It doesn’t \textit{do} anything, per se, it just warns when you accidentally use deprecated \LaTeX{} constructs from l2tabu (English / French / German / Italian / Spanish documentation).

From the documentation:

\fcolorbox{black}{yellow}{\parbox{\textwidth}{
		Old habits die hard. All the same, there are commands, classes and packages which are outdated and superseded. nag provides routines to warn the user about the use of those. As an example, we provide an extension that detects many of the “sins” described in l2tabu.
		}}

Therefore, I now always have the following in my header (\textit{before} the\lstinline[language={[LaTeX]TeX}]|documentclass|, thanks qbi):

\begin{lstlisting}[language={[LaTeX]TeX}]
\RequirePackage[l2tabu, orthodox]{nag}
\end{lstlisting}

It’s a bit like having \textbf{use strict}; in Perl: a useful best practice.

\section{tikz}
I nearly always use the \textbf{tikz} package. Once you learn how to draw with it, you can do almost any vector graphic you need.

\section{geometry}
\begin{lstlisting}[language={[LaTeX]TeX}]
\usepackage[width=21.00cm, height=29.70cm, left=2.54cm, right=2.54cm, top=2.54cm, bottom=2.54cm]{geometry}
\end{lstlisting}

\section{inputenc}
\begin{lstlisting}[language={[LaTeX]TeX}]
\usepackage[utf8]{inputenc}
\end{lstlisting}

\section{booktabs and array}
Another essential package combination is

\begin{lstlisting}[language={[LaTeX]TeX}]
\usepackage{booktabs}
\usepackage{array}
\end{lstlisting}

The \textbf{booktabs} package creates much nicer looking tables than the standard latex tables; the \textbf{array} package's ability to create custom columns is invaluable for formatting tabular material on a per-column basis.

\section{siunitx}
\textbf{siunitx}, for typesetting units and especially for the "S" column type, which allows numbers in tables to be easily aligned, e.g. on the decimal marker.

\begin{lstlisting}[language={[LaTeX]TeX}]
\usepackage{siunitx}
\end{lstlisting}

useful commands:

\begin{lstlisting}[language={[LaTeX]TeX}]
\numrange{1450}{1500}
\numrange[range-phrase = $ \sim $]{1450}{1500}
\numlist{0.1;0.2;0.3} \\
\numlist[list-separator = {; }]{0.1;0.2;0.3} \\
\numlist[list-final-separator = {, }]{0.1;0.2;0.3} \\
\numlist[list-pair-separator={/}]{300;220} % pay attention to the separator;

\SI[parse-numbers = false]{R}{\metre\per\second}
\SI[math-rm = \mathnormal, parse-numbers = false]{R}{\metre\per\second}
\SI[math-rm = \mathnormal,parse-numbers = false, per-mode=symbol]{n_1}{\rotation\per\minute}

$ n_1 = \SI[per-mode=symbol]{1450}{\rotation\per\minute} $
\end{lstlisting}

\section{mathtools}
In addition to many packages already listed here, I always include mathtools. It provides implementations of \lstinline[language={[LaTeX]TeX}]|mathclap| (and similar commands) as well as nice extensible arrow.

\begin{equation*}
\sum_{\mathclap{big long thing}}
\end{equation*}

\begin{lstlisting}[language={[LaTeX]TeX}]
\usepackage{mathtools}
\end{lstlisting}

\section{graphicx}
For including figures, rotating or scaling text. I also use the \lstinline[language={[LaTeX]TeX}]|graphicspath| command to specify a subfolder to help organize my figures and so I can easily change between, for example, a set of figures for internal used (with extra info) and final versions for distribution.

\begin{lstlisting}[language={[LaTeX]TeX}]
\usepackage{graphicx}
\graphicspath{{figures/}}
\end{lstlisting}

\section{listings}
I can't live without \textbf{listings} --- pretty-printing (colours, formatting and all) algorithms and code is indispensable --- in pretty much any programming languages and dialects under the sun. Plus, I can import a source file directly from the repository, and the latest version will be automatically rendered.

\begin{lstlisting}[language={[LaTeX]TeX}]
\usepackage{listings}
\end{lstlisting}

\section{tabularx}
I almost always find myself using a \textbf{tabularx} environment as opposed to the regular \textbf{tabular} environment, as it allows for greater dynamism in column widths.

\begin{lstlisting}[language={[LaTeX]TeX}]
\usepackage{tabularx} % tabularx environment. equivalent lenth?
\end{lstlisting}

\section{CJKutf8}
\textbf{CJKutf8} package is a part of \textbf{CJK} bundle, it is designed for documents in UTF-8 encoding only, and it actually loads \textbf{CJK} package internally.

The main aim of \textbf{CJKutf8} package is, to use utf8 option in inputenc package together with \textbf{CJK} package. That is to say, \textbf{CJKutf8} patches original \textbf{CJK} package to make it work well with inputenc. And it loads \textbf{inputenc} package with \textbf{utf8} option internally.

Most users do not need to know the technical details. But you can use \textbf{CJKutf8} to typeset French, German and Chinese in one document easily. That's it.

\begin{lstlisting}[language={[LaTeX]TeX}]
\usepackage{CJKutf8} % Chinese, Japanese, Korean input with utf-8 encoding. it loads \usepackage[utf8]{inputenc} internally
\end{lstlisting}

The usage is:

\begin{lstlisting}[language={[LaTeX]TeX}]
\begin{document}
	\begin{CJK*}{UTF8}{song}
	.
	.
	Contents...
	.
	.
	\end{CJK*}
\end{document}
\end{lstlisting}

\section{color}
Adding colors to your text is supported by the \textbf{color} package. Using this package, you can set the \textbf{font color}, \textbf{text background}, or \textbf{page background}. You can choose from \textbf{230 predefined colors} or can \textbf{define your own colors using RGB, Hex, or CMYK codes}. Mathematical formulas can also be colored.

\begin{lstlisting}[language={[LaTeX]TeX}]
\usepackage{color}
\end{lstlisting}

\section{longtable}

\begin{lstlisting}[language={[LaTeX]TeX}]
\usepackage{longtable}
\end{lstlisting}

\section{subcaption}

The usage of subcaption is:
\begin{lstlisting}[language={[LaTeX]TeX}]
\begin{figure}[htbp]
	\centering
	\begin{subfigure}[b]{0.45\textwidth}
		\centering
		\includegraphics[width=\textwidth]{dataDimensions}
		\caption{Data dimensions}\label{fig:subfig:datadim}
	\end{subfigure}
	\begin{subfigure}[b]{0.45\textwidth}
		\centering
		\includegraphics[width=\textwidth]{dataSize}
		\caption{Data Size}\label{fig:subfig:datasize}
	\end{subfigure}
	\caption{Scalability of data}
	\vspace{\baselineskip}
\end{figure}
\end{lstlisting}

\section{caption}

\begin{lstlisting}[language={[LaTeX]TeX}]
\begin{center}
	\includegraphics[width=0.4\textwidth]{XML}
	\captionof{figure}{Tree structure}\label{fig:xml_nonfloating}
	\vspace{\baselineskip}
\end{center}
\end{lstlisting}

\section{multirow}
\todo{multirow introductions}

%\section{fancyvrb}
%\todo{fancyvrb introductions}

\section{cleveref}
\subsection{Setup}
As always, the package is called in the preamble by writing \cverb|\usepackage{cleveref}|.In your document, all you have to do is write \cverb|\cref{...}| instead of figure~\cverb|\ref{...}|. The \textbf{cleveref} package will automatically detect where the reference is about and it will print the corresponding text (e.g. \textit{figure} when you refer to a figure environment).

\subsection{Capitalized references}
T capitalize your references, use \cverb|\Cref{...}| instead of \cverb|\cref{...}|.

\subsection{Multiple references}
Another thing that is quite nice is that \textbf{cleveref} allows multiple references with one command. For example, \cverb|\cref{eq1,eq2}| prints \cverb|‘eqs. (1) and (2)‘|. But that’s not all! When referencing to multiple environments (e.g. figures, equations and tables as in \cverb|\cref{fig1,eq3,tb1}|), the package automatically prints the right names to the right references.

\subsection{Editing the reference names}
I like my references to be written out in full, but the cleveref package prints ‘eq. …‘ instead of ‘equation …‘. Again, this is easily changed:

\begin{lstlisting}[language={[LaTeX]TeX}]
\crefname{equation}{equation}{equations}
\end{lstlisting}

In the first argument of \cverb|\crefname{}{}{}| comes the type of reference (equation, figure, table, section, etc.). The second argument contains the word that is printed if only one reference is made and the third argument contains the plural form for multiple references.

\begin{lstlisting}[language={[LaTeX]TeX}]
\cref{table: geometry factor of bevel gear}.
\end{lstlisting}

\todo{cleverref introductions}

\section{newverbs}
\begin{lstlisting}[language={[LaTeX]TeX}]
\usepackage{newverbs}
\newverbcommand{\cverb}{\color{red}}{} % colored vertb with red
\newverbcommand{\bverb}	% verbatim with gray background
{\begin{lrbox}{\verbbox}}
{\end{lrbox}\colorbox{gray}{\box\verbbox}}
\end{lstlisting}

Usage:
\begin{lstlisting}[language={[LaTeX]TeX}]
\cverb|&*/|
\bverb|&*/|
\end{lstlisting}

\section{natbib}
Usage:
\begin{lstlisting}[language={[LaTeX]TeX}]
\usepackage[numbers,sort]{natbib} % [1,3,2] => [1,2,3]
\usepackage[numbers,sort&compress]{natbib} % [1,3,2] => [1-3]
\end{lstlisting}

\section{Options that can be added to \textcolor{red}{\textbackslash usepackage}}
\begin{itemize}
\item \textcolor{red}{round}: (default) for round parentheses;
\item \textcolor{red}{square}: for square brackets;
\item \textcolor{red}{curly}: for curly braces;
\item \textcolor{red}{angle}: for angle brackets;
\item \textcolor{red}{colon}: (default) to separate multiple citations with colons;
\item \textcolor{red}{comma}: to use commas as separaters;
\item \textcolor{red}{authoryear}: (default) for author-year citations;
\item \textcolor{red}{numbers}: for numerical citations;
\item \textcolor{red}{super}: for superscripted numerical citations, as in \textit{Nature};
\item \textcolor{red}{sort}: orders multiple citations into  the sequence in which they appear in the list of references;
\item \textcolor{red}{sort\&compress}: as \textcolor{red}{sort}: but in addition multiple numerical citations are compressed if possible (as 3-6, 15)
\item \textcolor{red}{longnamesfirst}: makes the first citation of any reference the equivalent of starred variant (full author list) and subsequent citations normal (abbreviated list);
\item \textcolor{red}{sectionbib}: redefines \textcolor{red}{\textbackslash thebibliography} to issue \textcolor{red}{\textbackslash section*} instead of \textcolor{red}{\textbackslash chapter*}; valid only for classes with a \textcolor{red}{\textbackslash chapter} command; to be used with the \textcolor{red}{chapterbib} package;
\item \textcolor{red}{nonamebreak}: keeps all the authors' names in a citation on one line; causes overfull hboxes but helps with some \textcolor{red}{hyperref} problems.
\end{itemize}

\section{colortbl}
Use the \textbf{colortbl} package and its macro \lstinline[language={[LaTeX]TeX}]|\cellcolor{<color>}|

\section{enumerate}
Use the \lstinline[language={[LaTeX]TeX}]|\enumerate|

\begin{lstlisting}[language={[LaTeX]TeX}]
\begin{enumerate}[i]
\item One
\item Two
\item Three
\end{enumerate}
\end{lstlisting}

\section{platform package}

\begin{lstlisting}[language={[LaTeX]TeX}]
% packages we need
\usepackage{pdftexcmds}
\usepackage{catchfile}
\usepackage{ifluatex}
\usepackage{ifplatform}
\end{lstlisting}

\begin{lstlisting}[language={[LaTeX]TeX}]
\ifwindows
% add settings
\fi
\iflinux
% add settings
\fi
\ifmacosx
% add settings
\if
\end{lstlisting}

\section{InputIfFileExists}
\begin{lstlisting}[language={[LaTeX]TeX}]
\InputIfFileExists{"D:/Git/softwareLearning/Latex/LatexHelper.tex"}
{} % true condition.
{/Users/anzongzheng/Desktop/Git/softwareLearning/Latex/LatexHelper.tex} % false condition.
\end{lstlisting}
