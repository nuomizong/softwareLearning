% !Mode:: "TeX:UTF-8"	% read in as utf8 file.

\chapter{Font sizes, families, and styles}

\LaTeX normally choose the appropriate font and font size based on the logical structure of the document (e.g. sections). In some cases, you may want to set fonts and sizes by hand.

\section{Introduction}
The syntax to set a font size or font style is easy:

\begin{lstlisting}[language={[LaTeX]TeX}]
This is a simple example, {\tiny this will show different font sizes} and also \textsc{different font styles}.
\end{lstlisting}

This is a simple example, {\tiny this will show different font sizes} and also \textsc{different font styles}.

\section{Font sizes}
Font sizes are identified by special names, the actual size is not absolute but relative to the font size declared in the \lstinline[language={[LaTeX]TeX}]|\documentclass| statement.

\begin{lstlisting}[language={[LaTeX]TeX}]
In this example the {\huge huge font size} is set and the {\footnotesize Foot note size also}. There's a fairly large set of font sizes.
\end{lstlisting}

In this example the {\huge huge font size} is set and the {\footnotesize Foot note size also}. There's a fairly large set of font sizes.

In the example, \lstinline[language = {[LaTeX]TeX}]|{\huge huge font size}| declares that the next text inside the braces must be formatted in a \textit{huge} font size.

\section{Font families}
By default, in standard \LaTeX classes the serif typeface (a.k.a. roman) font is used. The other font typefaces (sans serif and typewriter, a.k.a. monospace) can be used by entering some specific commands.

\begin{lstlisting}[language={[LaTeX]TeX}]
In this example command and switches are used.
\texttt{A command is used to change the tyle of a setence}

\sffamily
A switch changes the style from this point to the end of the document unless other switch is used
\end{lstlisting}

In this example command and switches are used.
\texttt{A command is used to change the tyle of a setence}

{\sffamily
A switch changes the style from this point to the end of the document unless other switch is used}

You can set up the use of sans font as a default in \LaTeX document by using command:

\begin{lstlisting}[language={[LaTeX]TeX}]
\renewcommand{\familydefault}{\sfdefault}
\end{lstlisting}

Similarly, for using roman font as a default:

\begin{lstlisting}[language={[LaTeX]TeX}]
\renewcommand{\familydefault}{\rmdefault}
\end{lstlisting}

\section{Font styles}
The most common font styles in \LaTeX are bold, italics and underlined, but there are a few more.

\begin{lstlisting}[language={[LaTeX]TeX}]
Part of this text is written \textsl{in different font style} to highlight it.
\end{lstlisting}

Part of this text is written \textsl{in different font style} to highlight it.

In the example at the introduction the \textit{small caps} style was used. In this case the \lstinline[language={[LaTeX]TeX}]|\textsl| command sets the \textit{slanted} style which makes the text look a bit like \textit{italics} but not quite.

If you want to go back to "normal" font style (default for the \LaTeX class you are using), this can be done by using the \lstinline[{language=[LaTeX]TeX}]|\textnormal{...}| command or the \lstinline[language={[LaTeX]TeX}]|\normalfont| switch command.

\section{Reference sizes}

\begin{table}[h]
\caption{Font sizes}
\begin{tabular}{cc}
\hline \textbf{Command} & \textbf{Output} \\ 
\hline \lstinline[language={[LaTeX]TeX}]|\tiny| & \tiny Lorem ipsum \\ 
\hline \lstinline[language={[LaTeX]TeX}]|\scriptsize| & \scriptsize Lorem ipsum \\ 
\hline \lstinline[language={[LaTeX]TeX}]|\footnotesize| & \footnotesize Lorem ipsum \\ 
\hline \lstinline[language={[LaTeX]TeX}]|\small| & \small Lorem ipsum \\ 
\hline \lstinline[language={[LaTeX]TeX}]|\normalsize| & \normalsize Lorem ipsum \\ 
\hline \lstinline[language={[LaTeX]TeX}]|\large| & \large Lorem ipsum \\ 
\hline \lstinline[language={[LaTeX]TeX}]|\Large| & \Large Lorem ipsum \\ 
\hline \lstinline[language={[LaTeX]TeX}]|\LARGE| & \LARGE Lorem ipsum \\ 
\hline \lstinline[language={[LaTeX]TeX}]|\huge| & \huge Lorem ipsum \\ 
\hline \lstinline[language={[LaTeX]TeX}]|\Huge| & \Huge Lorem ipsum \\ 
\hline 
\end{tabular} 
\end{table}

\begin{table}[h]
\caption{Default font families}
\begin{tabular}{cccc}
\hline \textbf{typeface=family} & \textbf{command} & \textbf{switch command} & \textbf{output} \\ 
\hline serif(roman) & \lstinline[language={[LaTeX]TeX}]|\textrm{Sample Text 0123}| & \lstinline[language={[LaTeX]TeX}]|\rmfamily| & \textrm{Sample Text 0123} \\ 
\hline sans serif & \lstinline[language={[LaTeX]TeX}]|\textsf{Sample Text 0123}| & \lstinline[language={[LaTeX]TeX}]|\textsf| & \textsf{Sample Text 0123} \\ 
\hline typewriter(monospace) & \lstinline[language={[LaTeX]TeX}]|\texttt{Sample Text 0123}| & \lstinline[language={[LaTeX]TeX}]|\texttt| & \texttt{Sample Text 0123} \\ 
\hline 
\end{tabular} 
\end{table}

\begin{table}[h]
\caption{Font styles}
\begin{tabular}{cccc}
\hline \textbf{style} & \textbf{command} & \textbf{switch command} & \textbf{output} \\ 
\hline medium & \lstinline[language={[LaTeX]TeX}]|\textmd{Sample Text 0123}| & \lstinline[language={[LaTeX]TeX}]|\mdseries| & \textmd{Sample Text 0123} \\ 
\hline bold & \lstinline[language={[LaTeX]TeX}]|\textbf{Sample Text 0123}| & \lstinline[language={[LaTeX]TeX}]|\bfseries| & \textbf{Sample Text 0123} \\ 
\hline upright & \lstinline[language={[LaTeX]TeX}]|\textup{Sample Text 0123}| & \lstinline[language={[LaTeX]TeX}]|\upshape| & \textup{Sample Text 0123} \\ 
\hline italic & \lstinline[language={[LaTeX]TeX}]|\textit{Sample Text 0123}| & \lstinline[language={[LaTeX]TeX}]|\itshape| & \textit{Sample Text 0123} \\ 
\hline slanted & \lstinline[language={[LaTeX]TeX}]|\textsl{Sample Text 0123}| & \lstinline[language={[LaTeX]TeX}]|\slshape| & \textsl{Sample Text 0123} \\ 
\hline small caps & \lstinline[language={[LaTeX]TeX}]|\textsc{Sample Text 0123}| & \lstinline[language={[LaTeX]TeX}]|\scshape| & \textsc{Sample Text 0123} \\ 
\hline 
\end{tabular} 
\end{table}