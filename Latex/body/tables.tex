% !Mode:: "TeX:UTF-8"

\chapter{Table environment}
\section{Increase Matrix size}
\begin{lstlisting}[language={[LaTeX]TeX}]
\setcounter{MaxMatrixCols}{20}
\end{lstlisting}

\section{Complex table}
Using a combination of  \lstinline[language={[LaTeX]TeX}]|\multirow| and \lstinline[language={[LaTeX]TeX}]|\multicolumn| commands, We can create complex tables.

The \lstinline[language={[LaTeX]TeX}]|\multirow| command require \textbf{multirow} package. And the command is defined as follows:\\
\begin{lstlisting}[language={[LaTeX]TeX}]
	\multirow{nrows}{width}{content}
 \end{lstlisting}

Note that the \textbf{nrows} can be positive or negative, as illustrated in the following example.

\begin{table}
	\centering
	\begin{tabular}{|l|l|l|}
		\hline
		First & 		 \multicolumn{2}{c|}{Second} 		    \\
		\hline
		1 	  &   		     		  & \multirow{2}{80pt}{AB}  \\ \cline{1-1}
		2 	  &              		  & 						\\ \cline{1-1}	\cline{3-3}
		3 	  & \multirow{-3}{*}{123} & C						\\
		\hline
	\end{tabular}
\end{table}

\begin{table}[h]
	\caption{parameters for table} \label{tbl: parameters for table}
	\begin{tabular}{|l|l|}
		\hline
		l & left-justified column\\
		\hline
		c & centered column\\
		\hline
		r & right-justified column\\
		\hline
		p\{'width\}'& paragraph column with text vertically aligned at the top\\
		\hline
		m\{'width'\} & paragh column with text vertically aligned in the middle (requires array package)\\
		\hline
		b\{'width'\} & paragh column with text vertically aligned at the bottom (requires array package)\\
		\hline
		\textbar & vertical line \\
		\hline
		\textbardbl & double vertical line \\
		\hline
	\end{tabular}
\end{table}

\section{Three-line tables}
Three-line tables need~\textbf{booktabs} package~its standard format is shown in \cref{tab:table1}.

\begin{table}[htbp]
\caption{Three-line table specification}\label{tab:table1}
\vspace{0.5em}\centering
\begin{tabular}{ccccc}
\toprule[1.5pt]
$D$(in) & $P_u$(lbs) & $u_u$(in) & $\beta$ & $G_f$(psi.in)\\
\midrule[1pt]
 5 & 269.8 & 0.000674 & 1.79 & 0.04089\\
10 & 421.0 & 0.001035 & 3.59 & 0.04089\\
20 & 640.2 & 0.001565 & 7.18 & 0.04089\\
 5 & 269.8 & 0.000674 & 1.79 & 0.04089\\
10 & 421.0 & 0.001035 & 3.59 & 0.04089\\
20 & 640.2 & 0.001565 & 7.18 & 0.04089\\
 5 & 269.8 & 0.000674 & 1.79 & 0.04089\\
10 & 421.0 & 0.001035 & 3.59 & 0.04089\\
20 & 640.2 & 0.001565 & 7.18 & 0.04089\\
 5 & 269.8 & 0.000674 & 1.79 & 0.04089\\
10 & 421.0 & 0.001035 & 3.59 & 0.04089\\
20 & 640.2 & 0.001565 & 7.18 & 0.04089\\
\bottomrule[1.5pt]
\end{tabular}
\vspace{\baselineskip}
\end{table}

The original codes and explanation are:

\begin{lstlisting}[language={[LaTex]Tex}]
\begin{table}[htbp]
\caption{caption}\label{labelname(table: tablename)}
\vspace{0.5em}\centering
\begin{tabular}{cc...c}
\toprule[1.5pt]
The first cell of header  & The second cell of header   & ... & The nth cell of header  \\
\midrule[1pt]
cell(1,1) & cell(1,2) & ... & cell(1,n)\\
cell(2,1) & cell(2,2) & ... & cell(2,n)\\
cell(3,1) & cell(3,2) & ... & cell(3,n)\\
cell(4,1) & cell(4,2) & ... & cell(4,n)\\
...................................................\\
cell(m,1) & cell(m,2) & ... & cell(m,n)\\
\bottomrule[1.5pt]
\end{tabular}
\vspace{\baselineskip}
\end{table}
\end{lstlisting}

\section{Long table}
When you have a table that spans more than one page, the \textbf{longtable} package can help you out. It allows you to specify the column headings such that it prints on each page. Also, you can add a caption on each continued page to indicate that it the table is continued from the previous page. Similarly, you can add a footer to indicate that a table will be continued on the following page. Other than that, the longtable syntax is identical to the regular table environment. The following table spans more than one page:

\begin{center}
	\begin{longtable}{|l|l|l|}
		\caption[Feasible triples for a highly variable Grid]{Feasible triples for 
			highly variable Grid, MLMMH.} \label{grid_mlmmh} \\
		
		\hline
		\multicolumn{1}{|c|}{\textbf{Time (s)}} & \multicolumn{1}{c|}{\textbf{Triple chosen}} & \multicolumn{1}{c|}{\textbf{Other feasible triples}} \\
		\hline 
		\endfirsthead
		
		\multicolumn{3}{c}%
		{{\bfseries \tablename\ \thetable{} -- continued from previous page}} \\
		\hline
		\multicolumn{1}{|c|}{\textbf{Time (s)}} & \multicolumn{1}{c|}{\textbf{Triple chosen}} & \multicolumn{1}{c|}{\textbf{Other feasible triples}} \\
		\hline 
		\endhead
		
		\hline
		\multicolumn{3}{|r|}{{Continued on next page}} \\
		\hline
		\endfoot
		
		\hline \hline
		\endlastfoot
		
		0 & (1, 11, 13725) & (1, 12, 10980), (1, 13, 8235), (2, 2, 0), (3, 1, 0) \\
		2745 & (1, 12, 10980) & (1, 13, 8235), (2, 2, 0), (2, 3, 0), (3, 1, 0) \\
		5490 & (1, 12, 13725) & (2, 2, 2745), (2, 3, 0), (3, 1, 0) \\
		8235 & (1, 12, 16470) & (1, 13, 13725), (2, 2, 2745), (2, 3, 0), (3, 1, 0) \\
		10980 & (1, 12, 16470) & (1, 13, 13725), (2, 2, 2745), (2, 3, 0), (3, 1, 0) \\
		13725 & (1, 12, 16470) & (1, 13, 13725), (2, 2, 2745), (2, 3, 0), (3, 1, 0) \\
		16470 & (1, 13, 16470) & (2, 2, 2745), (2, 3, 0), (3, 1, 0) \\
		19215 & (1, 12, 16470) & (1, 13, 13725), (2, 2, 2745), (2, 3, 0), (3, 1, 0) \\
		21960 & (1, 12, 16470) & (1, 13, 13725), (2, 2, 2745), (2, 3, 0), (3, 1, 0) \\
		24705 & (1, 12, 16470) & (1, 13, 13725), (2, 2, 2745), (2, 3, 0), (3, 1, 0) \\
		27450 & (1, 12, 16470) & (1, 13, 13725), (2, 2, 2745), (2, 3, 0), (3, 1, 0) \\
		30195 & (2, 2, 2745) & (2, 3, 0), (3, 1, 0) \\
		32940 & (1, 13, 16470) & (2, 2, 2745), (2, 3, 0), (3, 1, 0) \\
		35685 & (1, 13, 13725) & (2, 2, 2745), (2, 3, 0), (3, 1, 0) \\
		38430 & (1, 13, 10980) & (2, 2, 2745), (2, 3, 0), (3, 1, 0) \\
		41175 & (1, 12, 13725) & (1, 13, 10980), (2, 2, 2745), (2, 3, 0), (3, 1, 0) \\
		43920 & (1, 13, 10980) & (2, 2, 2745), (2, 3, 0), (3, 1, 0) \\
		46665 & (2, 2, 2745) & (2, 3, 0), (3, 1, 0) \\
		49410 & (2, 2, 2745) & (2, 3, 0), (3, 1, 0) \\
		52155 & (1, 12, 16470) & (1, 13, 13725), (2, 2, 2745), (2, 3, 0), (3, 1, 0) \\
		54900 & (1, 13, 13725) & (2, 2, 2745), (2, 3, 0), (3, 1, 0) \\
		57645 & (1, 13, 13725) & (2, 2, 2745), (2, 3, 0), (3, 1, 0) \\
		60390 & (1, 12, 13725) & (2, 2, 2745), (2, 3, 0), (3, 1, 0) \\
		63135 & (1, 13, 16470) & (2, 2, 2745), (2, 3, 0), (3, 1, 0) \\
		65880 & (1, 13, 16470) & (2, 2, 2745), (2, 3, 0), (3, 1, 0) \\
		68625 & (2, 2, 2745) & (2, 3, 0), (3, 1, 0) \\
		71370 & (1, 13, 13725) & (2, 2, 2745), (2, 3, 0), (3, 1, 0) \\
		74115 & (1, 12, 13725) & (2, 2, 2745), (2, 3, 0), (3, 1, 0) \\
		76860 & (1, 13, 13725) & (2, 2, 2745), (2, 3, 0), (3, 1, 0) \\
		79605 & (1, 13, 13725) & (2, 2, 2745), (2, 3, 0), (3, 1, 0) \\
		82350 & (1, 12, 13725) & (2, 2, 2745), (2, 3, 0), (3, 1, 0) \\
		85095 & (1, 12, 13725) & (1, 13, 10980), (2, 2, 2745), (2, 3, 0), (3, 1, 0) \\
		87840 & (1, 13, 16470) & (2, 2, 2745), (2, 3, 0), (3, 1, 0) \\
		90585 & (1, 13, 16470) & (2, 2, 2745), (2, 3, 0), (3, 1, 0) \\
		93330 & (1, 13, 13725) & (2, 2, 2745), (2, 3, 0), (3, 1, 0) \\
		96075 & (1, 13, 16470) & (2, 2, 2745), (2, 3, 0), (3, 1, 0) \\
		98820 & (1, 13, 16470) & (2, 2, 2745), (2, 3, 0), (3, 1, 0) \\
		101565 & (1, 13, 13725) & (2, 2, 2745), (2, 3, 0), (3, 1, 0) \\
		104310 & (1, 13, 16470) & (2, 2, 2745), (2, 3, 0), (3, 1, 0) \\
		107055 & (1, 13, 13725) & (2, 2, 2745), (2, 3, 0), (3, 1, 0) \\
		109800 & (1, 13, 13725) & (2, 2, 2745), (2, 3, 0), (3, 1, 0) \\
		112545 & (1, 12, 16470) & (1, 13, 13725), (2, 2, 2745), (2, 3, 0), (3, 1, 0) \\
		115290 & (1, 13, 16470) & (2, 2, 2745), (2, 3, 0), (3, 1, 0) \\
		118035 & (1, 13, 13725) & (2, 2, 2745), (2, 3, 0), (3, 1, 0) \\
		120780 & (1, 13, 16470) & (2, 2, 2745), (2, 3, 0), (3, 1, 0) \\
		123525 & (1, 13, 13725) & (2, 2, 2745), (2, 3, 0), (3, 1, 0) \\
		126270 & (1, 12, 16470) & (1, 13, 13725), (2, 2, 2745), (2, 3, 0), (3, 1, 0) \\
		129015 & (2, 2, 2745) & (2, 3, 0), (3, 1, 0) \\
		131760 & (2, 2, 2745) & (2, 3, 0), (3, 1, 0) \\
		134505 & (1, 13, 16470) & (2, 2, 2745), (2, 3, 0), (3, 1, 0) \\
		137250 & (1, 13, 13725) & (2, 2, 2745), (2, 3, 0), (3, 1, 0) \\
		139995 & (2, 2, 2745) & (2, 3, 0), (3, 1, 0) \\
		142740 & (2, 2, 2745) & (2, 3, 0), (3, 1, 0) \\
		145485 & (1, 12, 16470) & (1, 13, 13725), (2, 2, 2745), (2, 3, 0), (3, 1, 0) \\
		148230 & (2, 2, 2745) & (2, 3, 0), (3, 1, 0) \\
		150975 & (1, 13, 16470) & (2, 2, 2745), (2, 3, 0), (3, 1, 0) \\
		153720 & (1, 12, 13725) & (2, 2, 2745), (2, 3, 0), (3, 1, 0) \\
		156465 & (1, 13, 13725) & (2, 2, 2745), (2, 3, 0), (3, 1, 0) \\
		159210 & (1, 13, 13725) & (2, 2, 2745), (2, 3, 0), (3, 1, 0) \\
		161955 & (1, 13, 16470) & (2, 2, 2745), (2, 3, 0), (3, 1, 0) \\
		164700 & (1, 13, 13725) & (2, 2, 2745), (2, 3, 0), (3, 1, 0) \\
	\end{longtable}
\end{center}

\noindent\hrule
\section{列宽可调表格的绘制方法}
论文中能用到列宽可调表格的情况共有两种:一种是当插入的表格某一单元格内容过长以至于一行放不下的情况,
另一种是当对公式中首次出现的物理量符号进行注释的情况。这两种情况都需要调用~tabularx~宏包。下面将分别对这两种情况下可调表格的绘制方法进行阐述。
\section{表格内某单元格内容过长的情况}

首先给出这种情况下的一个例子如表~\ref{tab:table3}~所示。
\begin{table}[htbp]
	\caption{最小的三个正整数的英文表示法}\label{tab:table3}
	\vspace{0.5em}
	\begin{tabularx}{\textwidth}{llX}
		\toprule[1.5pt]
		Value & Name & Alternate names, and names for sets of the given size\\\midrule[1pt]
		1 & One & ace, single, singleton, unary, unit, unity\\
		2 & Two & binary, brace, couple, couplet, distich, deuce, double, doubleton, duad, duality, duet, duo, dyad, pair, snake eyes, span, twain, twosome, yoke\\
		3 & Three & deuce-ace, leash, set, tercet, ternary, ternion, terzetto, threesome, tierce, trey, triad, trine, trinity, trio, triplet, troika, hat-trick\\\bottomrule[1.5pt]
	\end{tabularx}
	\vspace{\baselineskip}
\end{table}
绘制这种表格的代码及其说明如下。

\vspace{1em}\noindent\hrule
\begin{verbatim}
\begin{table}[htbp]
\caption{表标题}\label{标签名(通常为 tab:tablename)}
\vspace{0.5em}\wuhao
\begin{tabularx}{\textwidth}{l...X...l}
\toprule[1.5pt]
表头第1个格   & ... & 表头第X个格   & ... & 表头第n个格  \\
\midrule[1pt]
表中数据(1,1) & ... & 表中数据(1,X) & ... & 表中数据(1,n)\\
表中数据(2,1) & ... & 表中数据(2,X) & ... & 表中数据(2,n)\\
.........................................................\\
表中数据(m,1) & ... & 表中数据(m,X) & ... & 表中数据(m,n)\\
\bottomrule[1.5pt]
\end{tabularx}
\vspace{\baselineskip}
\end{table}
\end{verbatim}

\noindent\hrule
\begin{verbatim}
tabularx环境共有两个必选参数:第1个参数用来确定表格的总宽度,这里取为排版表格能达到的最大宽度——正文宽度\textwidth;第2 个参数用来确定每列格式,其中标为X的项表示该列的宽度可调,其宽度值由表格总宽度确定。
标为X的列一般选为单元格内容过长而无法置于一行的列,这样使得该列内容能够根据表格总宽度自动分行。若列格式中存在不止一个X 项,则这些标为X的列的列宽相同,因此,一般不将内容较短的列设为X。
标为X的列均为左对齐,因此其余列一般选为l(左对齐),这样可使得表格美观,但也可以选为c或r。
\end{verbatim}

\noindent\hrule
\section{对物理量符号进行注释的情况}
为使得对公式中物理量符号注释的转行与破折号“———”后第一个字对齐,此处最好采用表格环境。此表格无任何线条,左对齐,
且在破折号处对齐,一共有“式中”二字、物理量符号和注释三列,表格的总宽度可选为文本宽度,因此应该采用\verb|tabularx|环境。
由\verb|tabularx|环境生成的对公式中物理量符号进行注释的公式如式(\ref{eq:1})所示。
%\vspace*{10pt}

\begin{equation}\label{eq:1}
\ddot{\boldsymbol{\rho}}-\frac{\mu}{R_{t}^{3}}\left(3\mathbf{R_{t}}\frac{\mathbf{R_{t}\rho}}{R_{t}^{2}}-\boldsymbol{\rho}\right)=\mathbf{a}
\end{equation}

%\begin{tabularx}{\textwidth}{@{}l@{\quad}r@{———}X@{}}
%	式中& $\bm{\rho}$ &追踪飞行器与目标飞行器之间的相对位置矢量;\\
%	&  $\bm{\ddot{\rho}}$&追踪飞行器与目标飞行器之间的相对加速度;\\
%	&  $\mathbf{a}$   &推力所产生的加速度;\\
%	&  $\mathbf{R_t}$ & 目标飞行器在惯性坐标系中的位置矢量;\\
%	&  $\omega_{t}$ & 目标飞行器的轨道角速度;\\
%	&  $\mathbf{g}$ & 重力加速度,$=\frac{\mu}{R_{t}^{3}}\left(
%	3\mathbf{R_{t}}\frac{\mathbf{R_{t}\rho}}{R_{t}^{2}}-\bm{\rho}\right)=\omega_{t}^{2}\frac{R_{t}}{p}\left(
%	3\mathbf{R_{t}}\frac{\mathbf{R_{t}\rho}}{R_{t}^{2}}-\bm{\rho}\right)$,这里~$p$~是目标飞行器的轨道半通径。
%\end{tabularx}
%\vspace{\wordsep}

其中生成注释部分的代码及其说明如下。

\vspace{1em}\noindent\hrule

\begin{verbatim}
\begin{tabularx}{\textwidth}{@{}l@{\quad}r@{— — —}X@{}}
式中 & symbol-1 & symbol-1的注释内容;\\
& symbol-2 & symbol-2的注释内容;\\
.............................;\\
& symbol-m & symbol-m的注释内容。
\end{tabularx}\vspace{\wordsep}
\end{verbatim}

\noindent\hrule

\begin{verbatim}
tabularx环境的第1个参数选为正文宽度,第2个参数里面各个符号的意义为:
第1个@{}表示在“式中”二字左侧不插入任何文本,“式中”二字能够在正文中左对齐,若无此项,则“式中”二字左侧会留出一定的空白;
@{\quad}表示在“式中”和物理量符号间插入一个空铅宽度的空白;
@{— — —}实现插入破折号的功能,它由三个1/2的中文破折号构成;
第2个@{}表示在注释内容靠近正文右边界的地方能够实现右对齐。
\end{verbatim}

\noindent\hrule\vspace{1em}

由此方法生成的注释内容应紧邻待注释公式并置于其下方,因此不能将代码放入~\verb|table|~浮动环境中。但此方法不能实现自动转页接排,
可能会在当前页剩余空间不够时,全部移动到下一页而导致当前页出现很大空白。因此在需要转页处理时,还请您手动将需要转页的代码放入一个
新的~\verb|tabularx|~环境中,将原来的一个~\verb|tabularx|~环境拆分为两个~\verb|tabularx|~环境。

若想获得绘制表格的更多信息,参见网络上的~\href{http://www.tug.org/pracjourn/2007-1/mori/}{Tables in \LaTeXe: Packages and Methods}~文档。

\section{Remove indentations}
\subsection{Indentation of whole table}
Simply add \textbf{\textbackslash noindent} before \textbackslash begin{table} environment.
\subsection{Indentation within table}
Remove the (column separating) space on the left (and possibly on the right) side with a slight modification of the table head. For example:

\begin{tabular}{@{} ll @{}}
	first & second\\
	first & second\\
\end{tabular}

\begin{lstlisting}[language={[LaTeX]TeX}]
\begin{tabular}{@{} ll @{}}
first & second\\
first & second\\
\end{tabular}
\end{lstlisting}

\section{Colored table}
\begin{tabular}{l|c|r}
\hline
some & \cellcolor{green}coloured & contents \\
\hline
\end{tabular}

\begin{lstlisting}[language={[LaTeX]TeX}]
\begin{tabular}{l|c|r}
\hline
some & \cellcolor{green}coloured & contents \\
\hline
\end{tabular}
\end{lstlisting}