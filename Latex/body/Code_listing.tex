% !Mode:: "TeX:UTF-8"

\chapter{Code listing}

\LaTeX{} is widely used in science and programming has become an important aspect in several areas of science, hence the need for a tool that properly displays code. In this article is explained how to use the standard \textbf{verbatim} environment as well as the package \textbf{listings}, which provide more advanced code formatting features.

\section{Introduction}
Displaying code in \LaTeX{}is straightforward. For instance, using the \textbf{lstlisting} environment:

\begin{lstlisting}[language=python]
import numpy as np

def incmatrix(genl1,genl2):
	m = len(genl1)
	n = len(genl2)
	M = None #to become the incidence matrix
	VT = np.zeros((n*m,1), int)  #dummy variable
	
	#compute the bitwise xor matrix
	M1 = bitxormatrix(genl1)
	M2 = np.triu(bitxormatrix(genl2),1) 
	
	for i in range(m-1):
		for j in range(i+1, m):
			[r,c] = np.where(M2 == M1[i,j])
				for k in range(len(r)):
					VT[(i)*n + r[k]] = 1;
					VT[(i)*n + c[k]] = 1;
					VT[(j)*n + r[k]] = 1;
					VT[(j)*n + c[k]] = 1;
					
					if M is None:
						M = np.copy(VT)
					else:
						M = np.concatenate((M, VT), 1)
					
					VT = np.zeros((n*m,1), int)
	
	return M
\end{lstlisting}


In this example, the outupt ignores all \LaTeX{} commands and the text is printed keeping all the line breaks and white spaces typed. To use the \textit{\bfseries lstlisting} environment you have to add the next line to the preamble of your document:

\fbox{\textbackslash usepackage\{\textcolor{blue}{listings}\}}

\section{The verbatim environment}
The default tool to display code in \LaTeX{} is verbatim, which generates an output in monospaced font.

\vspace{1em} \noindent \hrule
\begin{verbatim}
Text enclosed inside \texttt{verbatim} environment 
is printed directly 
and all \LaTeX{} commands are ignored.
\end{verbatim}
\noindent \hrule \vspace{1em}

Just as in the example at the introduction, all text is printed keeping line breaks and white spaces. There's a starred version of this command whose output is slightly different.

\vspace{1em} \noindent \hrule
\begin{verbatim*}
Text enclosed inside \texttt{verbatim} environment 
is printed directly 
and all \LaTeX{} commands are ignored.
\end{verbatim*}
\noindent \hrule \vspace{1em} 

In this case white spaces are emphasized with a special symbol.

Verbatim-like text can also be used in a paragraph by means of the \textbackslash verb command.

\vspace{1em} \noindent \hrule
In the directory \verb|C:\Windows\system32| you can find a lot of Windows 
system applications. 

The \verb+\ldots+ command produces \ldots
\noindent \hrule \vspace{1em} 

The command \textbackslash verb \textbar \verb|C:\Windows\system32| \textbar~prints the text inside the delimiters \textbar in verbatim format. Any character, except letters and *, can be used as delimiter. For instance \textbackslash verb+\ldots+ uses + as delimiter.

\section{Using listings to highlight code}
In the introduction a basic example of the package \textbf{listings} was presented, let's see a second example:

\begin{lstlisting}[language=Python]
import numpy as np

def incmatrix(genl1,genl2):
	m = len(genl1)
	n = len(genl2)
	M = None #to become the incidence matrix
	VT = np.zeros((n*m,1), int)  #dummy variable
	
	#compute the bitwise xor matrix
	M1 = bitxormatrix(genl1)
	M2 = np.triu(bitxormatrix(genl2),1) 
	
	for i in range(m-1):
		for j in range(i+1, m):
			[r,c] = np.where(M2 == M1[i,j])
			for k in range(len(r)):
				VT[(i)*n + r[k]] = 1;
				VT[(i)*n + c[k]] = 1;
				VT[(j)*n + r[k]] = 1;
				VT[(j)*n + c[k]] = 1;
				
				if M is None:
					M = np.copy(VT)
				else:
					M = np.concatenate((M, VT), 1)
				
				VT = np.zeros((n*m,1), int)

	return M
\end{lstlisting}

The additional parameter inside brackets [language=Python] enables code highlighting for this particular programming language (Python), special words are in boldface font and comments are italicized. See the reference guide for a complete list of supported programming languages.

\section{Importing code from a file}
Code is usually stored in a source file, therefore a command that automatically pulls code from a file becomes very handy.

\fbox{\parbox{\textwidth}{
		The next code will be directly imported from a file \\
		
		\textbackslash lstinputlisting[language=Octave]\{BitXorMatrix.m\}
		}}
		
The command \textbackslash lstinputlisting[language=Octave]\{BitXorMatrix.m\} imports the code from the file \textit{BitXorMatrix.m}, the additional parameter in between brackets enables language highlighting for the Octave programming language. If you need to import only part of the file you can specify two comma-separated parameters inside the brackets. For instance, to import the code from the line 2 to the line 12, the previous command becomes

\fbox{\textbackslash lstinputlisting[language=Octave, firstline=2, lastline=12]\{BitXorMatrix.m\}}

If firstline or lastline is omitted, it's assumed that the values are the beginning of the file, or the bottom of the file, respectively.

\section{Code styles and colours}
Code formatting with the \textbf{listing} package is highly customisable. Let's see an example

\begin{lstlisting}[language={[LaTeX]TeX}, caption=Code styles and colours]
\documentclass{article}
\usepackage[utf8]{inputenc}

\usepackage{listings}
\usepackage{color}

\definecolor{codegreen}{rgb}{0,0.6,0}
\definecolor{codegray}{rgb}{0.5,0.5,0.5}
\definecolor{codepurple}{rgb}{0.58,0,0.82}
\definecolor{backcolour}{rgb}{0.95,0.95,0.92}

\lstdefinestyle{zstyle}{
	basicstyle=\small\tt,
	backgroundcolor=\color{backcolour},   
	commentstyle=\color{codegreen},
	keywordstyle=\color{magenta},
	numberstyle=\tiny\color{codegray},
	stringstyle=\color{codepurple},
	%basicstyle=\footnotesize,
	breakatwhitespace=false,         
	breaklines=true,                 
	captionpos=b,                    
	keepspaces=true,                 
	numbers=left,                    
	numbersep=5pt,                  
	showspaces=false,                
	showstringspaces=false,
	showtabs=false,                  
	tabsize=2
}

\lstset{style=zstyle}

\begin{document}
The next code will be directly imported from a file

\lstinputlisting[language=Octave]{BitXorMatrix.m}
\end{document}
\end{lstlisting}

As you see, the code colouring and styling greatly improves readability.

In this example the package \textbf{color} is imported and then the command \textbackslash definecolor\{\}\{\}\{\} is used to define new colours in rgb format that will later be used. The package \textbf{xcolor} also works for this. For more information see: \underline{using colours in \LaTeX}

There are essentially two commands that generate the style for this example:

\fbox{\textbackslash lstdefinestyle\{mystyle\}\{\ldots \}}

Defines a new code listing style called "mystyle". Inside the second pair of braces the options that define this style are passed; see the reference guide for a full description of these and some other parameters.

\fbox{\textbackslash lstset\{style=mystyle\} }

Enables the style "mystyle". This command can be used within your document to switch to a different style if needed.

\section{Captions and the list of Listings}
Just like in floats (\textcolor{red}{tables} and \textcolor{red}{figures}), captions can be added to listing for a more clear presentation.

\begin{lstlisting}[language=Python, morekeywords={as}, caption=Python example]
import numpy as np

	def incmatrix(genl1,genl2):
	m = len(genl1)
	n = len(genl2)
	M = None #to become the incidence matrix
	VT = np.zeros((n*m,1), int)  #dummy variable
	
	#compute the bitwise xor matrix
	M1 = bitxormatrix(genl1)
	M2 = np.triu(bitxormatrix(genl2),1) 
	
	for i in range(m-1):
		for j in range(i+1, m):
			[r,c] = np.where(M2 == M1[i,j])
			for k in range(len(r)):
				VT[(i)*n + r[k]] = 1;
				VT[(i)*n + c[k]] = 1;
				VT[(j)*n + r[k]] = 1;
				VT[(j)*n + c[k]] = 1;
				
				if M is None:
					M = np.copy(VT)
				else:
					M = np.concatenate((M, VT), 1)
				
				VT = np.zeros((n*m,1), int)
	
	return M
\end{lstlisting}

Adding the comma-separated parameter caption=Python example inside the brackets, enables the caption. This caption can be later used in the list of Listings.

\fbox{\textbackslash lstlistoflistings}
%\lstlistoflistings

\section{Reference guide}
Supported languages

\textbf{supported languages (and its dialects if possible, dialects are specified in brackets and default dialects are italized)}:

\begin{center}
	\begin{longtable}{l|l}
		\caption{supported languages of listings} \label{longtable: supported languages of listings} \\
		ABAP (R/2 4.3, R/2 5.0, R/3 3.1, R/3 4.6C, R/3 6.10)	& ACSL \\
		\hline
		Ada (2005, 83, 95)	& Algol (60, 68)\\
		\hline
		Ant	& Assembler (Motorola68k, x86masm)\\
		\hline
		Awk (gnu, POSIX)	& bash\\
		\hline
		Basic (Visual)	& C (ANSI, Handel, Objective, Sharp)\\
		\hline
		C++ (ANSI, GNU, ISO, Visual)	& Caml (light, Objective)\\
		\hline
		CIL	& Clean\\
		\hline
		Cobol (1974, 1985, ibm)	& Comal 80 \\
		\hline
		command.com (WinXP)	& Comsol  \\
		\hline
		csh	& Delphi  \\
		\hline
		Eiffel	& Elan  \\
		\hline
		erlang	& Euphoria\\
		\hline
		Fortran (77, 90, 95)	& GCL\\
		\hline
		Gnuplot	& Haskell\\
		\hline
		HTML	& IDL (empty, CORBA)\\
		\hline
		inform	& Java (empty, AspectJ)\\
		\hline
		JVMIS	& ksh\\
		\hline
		Lingo	& Lisp (empty, Auto)\\
		\hline
		Logo	& make (empty, gnu)\\
		\hline
		Mathematica (1.0, 3.0, 5.2)	& Matlab\\
		\hline
		Mercury	& MetaPost\\
		\hline
		Miranda	& Mizar\\
		\hline
		ML	& Modula-2\\
		\hline
		MuPAD	& NASTRAN\\
		\hline
		Oberon-2	& OCL (decorative, OMG)\\
		\hline
		Octave	& Oz\\
		\hline
		Pascal (Borland6, Standard, XSC)	& Perl\\
		\hline
		PHP	& PL/I\\
		\hline
		Plasm	& PostScript\\
		\hline
		POV	& Prolog\\
		\hline
		Promela	& PSTricks\\
		\hline
		Python	& R\\
		\hline
		Reduce	& Rexx\\
		\hline
		RSL	& Ruby\\
		\hline
		S (empty, PLUS)	& SAS\\
		\hline
		Scilab	& sh\\
		\hline
		SHELXL	& Simula (67, CII, DEC, IBM)\\
		\hline
		SPARQL	& SQL\\
		\hline
		tcl (empty, tk)	& TeX (AlLaTeX, common, LaTeX, plain, primitive)\\
		\hline
		VBScript	& Verilog\\
		\hline
		VHDL (empty, AMS)	& VRML (97)\\
		\hline
		XML	& XSLT\\
		\hline
	\end{longtable}
\end{center}

\subsection{Options to customize code listing styles}
\begin{itemize}
	\item \textbf{backgroundcolor} - colour for the background. External \textit{\bfseries color} or \textit{\bfseries xcolor} package needed.
	\item \textbf{commentstyle} - style of comments in source language.
	\item \textbf{basicstyle} - font size/family/etc. for source (e.g. basicstyle=\textbackslash ttfamily\textbackslash small)
	\item \textbf{keywordstyle} - style of keywords in source language (e.g. keywordstyle=\textbackslash color \{red\})
	\item \textbf{numberstyle} - style used for line-numbers
	\item \textbf{numberserp} - distance of line-numbers from the code
	\item \textbf{stringstyle} - style of strings in source language
	\item \textbf{showspaces} - emphasize spaces in code (true/false)
	\item \textbf{showstringspaces} - emphasize spaces in strings (true/false)
	\item \textbf{showtabs} - emphasize tabulators in code (true/false)
	\item \textbf{numbers} - position of line numbers (left/right/none, i.e. no line numbers)
	\item \textbf{prebreak} - displaying mark on the end of breaking line (e.g. prebreak=\textbackslash raisebox\{0ex\}[0ex][0ex]\{\textbackslash ensuremath\{\textbackslash hookleftarrow\}\})
	\item \textbf{captionpos} - position of caption (t/b)
	\item \textbf{frame} - showing frame outside code (none/leftline/topline/bottomline/lines/single/shadowbox)
	\item \textbf{breakwhitespace} - sets if automatic breaks should only happen at whitespaces
	\item \textbf{breaklines} - automatic line-breaking
	\item \textbf{keepspaces} - keep spaces in the code, useful for indetation
		tabsize - default tabsize
	\item \textbf{escapeinside} - specify characters to escape from source code to LATEX (e.g. escapeinside=\{\%*\}\{*)\})
	\item \textbf{rulecolor} - Specify the colour of the frame-box
\end{itemize}

\section{Add more keywords}
Keywords are closely related to languages, so it should not be specified when setting language style. It should be used like:

\begin{lstlisting}[language={[LaTeX]TeX}]
\begin{lstlisting}[language=Python, morekeywords={as}, caption=Python example]
\end{lstlisting}

Or if one language is extensively used, then do like:

\begin{lstlisting}[language={[LaTeX]TeX}]
\lstset{style=zstyle, language=python, morekeywords={as}}
\begin{lstlisting}
import numpy as np

def incmatrix(genl1,genl2):
	m = len(genl1)
	n = len(genl2)
	M = None #to become the incidence matrix
	VT = np.zeros((n*m,1), int)  #dummy variable
	
	#compute the bitwise xor matrix
	M1 = bitxormatrix(genl1)
	M2 = np.triu(bitxormatrix(genl2),1) 
	
	for i in range(m-1):
		for j in range(i+1, m):
		[r,c] = np.where(M2 == M1[i,j])
			for k in range(len(r)):
				VT[(i)*n + r[k]] = 1;
				VT[(i)*n + c[k]] = 1;
				VT[(j)*n + r[k]] = 1;
				VT[(j)*n + c[k]] = 1;
				
				if M is None:
					M = np.copy(VT)
				else:
					M = np.concatenate((M, VT), 1)
				
				VT = np.zeros((n*m,1), int)

	return M
\end{lstlisting}
