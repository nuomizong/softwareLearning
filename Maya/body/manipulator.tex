% !Mode:: "TeX:UTF-8"	% read in as utf8 file.

\chapter{Manipulator}

\section{Basic operations}

\subsection{Move, Rotate and Scale view port}

\subsubsection{Windows}
\begin{enumerate}
	\item Move
	The \textbf{Move} operation is done by hold \textcolor{red}{ALT} and \textcolor{red}{Middle} button;
	\item Rotate
	The \textbf{Rotate} operation is done by holding \textcolor{red}{ALT} and \textcolor{red}{Left} button;
	\item Scale
	The \textbf{Scale} operation is done by holding \textcolor{red}{ALT} and \textcolor{red}{Right} button.
	\item[pivot] press \textcolor{red}{Insert}
\end{enumerate}

\subsubsection{Mac}
\lstinline{Maya>Preference>Interface>device}
\begin{enumerate}
	\item roll mouse: deactivate
	\item track mouse: one button
	\item multi-touch gesture: deactivate
	\item touchpad: cursor control
\end{enumerate}

\begin{enumerate}
\item[pivot] press \textcolor{red}{d}
\end{enumerate}
option + mouse = rotate.
option + control + mouse = scale.
option + command+mouse = move.

option + force pad down = rotate
option + force pad down up or downwards = scale
option + alt = move

\subsubsection{oversized}
Use \textcolor{red}{+/-} keys in keyboard to increase/decrease arrows' length 

\subsection{Change view port numbers}
Four views and one view can be achieved simply by press \textcolor{red}{Space Bar}.

\subsection{Outliner - Objects list and selection}
To see a list of all these objects and pick one easily, the \textcolor{red}{Outliner} is used. It is in the \textcolor{red}{Window Menu}.

\subsection{Change polygon subdivisions}
\textcolor{red}{Subdivisions} attributes of a polygon is lied in the \textcolor{red}{PolyXXX attribute}. Normally, there are three options: \textcolor{red}{subdivisions Axis}, \textcolor{red}{Subdivisions Height}, \textcolor{red}{Subdivisions Caps}.

\subsection{Add divisions}
Alternate to\textcolor{red}{changing polygon subdivisions}, \textcolor{red}{subdivisions of faces} can be changed directly using \textcolor{red}{Add divisions} options in the \textcolor{red}{Edit mesh} menu.

\subsection{Hide and display objects}
Select \textcolor{red}{objects} in the \textcolor{red}{Outliner} and use short key \textcolor{red}{Ctrl + h} to \textbf{hide} selected objects. Contrary, use \textcolor{red}{Shift + h} to \textbf{Show} selected objects.

\subsection{Delete extra vertices, edges}
Holding on \textcolor{red}{Left button} on the mouse to change to select \textcolor{red}{Vertice or Edge mode}.\\

If vertices are deleted, the corresponding edges will be rebuilt. This operation has an effect on edges.\\

If edges are deleted, the related faces will be reconstructed. This operation affects the division of faces. So can be adopted to eliminate extra divisions.

\subsection{Using script editor}
\textcolor{red}{Script Editor} is a powerful tool to see all former operations in a form of \textcolor{red}{Mel language}. Meanwhile, it contributes to find out:\\
\begin{enumerate}
	\item How many items, such as vertices, edges or faces etc, are selected.
	\item the indices of selected items.
	\item check operations and selections.
\end{enumerate}

\subsection{Move the centroid}
Select a object or component, then choose \textcolor{red}{Move, Rotate or scale operation}, after that, press \textcolor{red}{Insert} to manipulate the \textcolor{red}{Pivot}.

\subsection{Select specific vertex, edge or facet}
\begin{itemize}
	\item select one vertex
	\textbf{select -r Mesh.vtx[1]}.
	\item select one edge
	\textbf{select -r Mesh.vtx[1]}.
	\item Select one facet
	\textbf{select -r Mesh.e[1]}.
	\item Select several facet
	\textbf{select -r Mesh.f[1]}.
	\textbf{select -tgl Mesh.f[2]}.
	\textbf{select -tgl Mesh.f[5]}.
\end{itemize}

\subsection{Snap / Move vertex to another vertex}
In MAYA, change to \textbf{Vertex mode}, and go to the \textbf{Tool menu} and click \textbf{Snap to points} then use \textbf{Move operator} to \textbf{Snap this vertex to another vertex}.

\subsection{Snap / Move pivot to specific vertex}
After \textbf{Extrusion}, the pivot has been changed and if we move the \textbf{extrusion} right now, the profile will deform. So must make sure the pivot is in right place.

\subsection{Duplicate objects}
The easiest way to duplicate object is select \textbf{Move operator} then press \textbf{Ctrl+D}. After that, \textbf{move operator} changes to \textbf{Duplicate operator}. to exit \textbf{Duplicate operator}, press \textbf{d}.

\subsection{Connect vertices}
\textbf{Connect vertices} will connect two vertices to create a new edge. This tool locates in \textbf{Top-right corner} named \textbf{Show/Hide Modeling toolkit}.

\subsection{Select visible vertices, edges and faces}
You can use \textbf{backface culling}, it's under \textbf{Display-\textless Polygon} or you can open the tool settings for any transformation tool (select, move, scale etc) and check \textbf{camera based selection}. The two methods are shown in figure~\ref{fig: backface culling} and figure~\ref{fig:camera based selection}.

\begin{figure}[htb]
	\centering
	\begin{minipage}[b]{0.45\textwidth}
		\includegraphics[width=\textwidth]{"figures/backface culling"}
		\caption{backface culling}\label{fig: backface culling}
	\end{minipage}
	\begin{minipage}[b]{0.45\textwidth}
		\centering
		\includegraphics[width=\textwidth]{"figures/camera based selection"}
		\caption{camera based selection}\label{fig:camera based selection}
	\end{minipage}
	\vspace{\baselineskip}
\end{figure}

\subsection{View component information}
Polygon elements, such as facets, edges are composed of vertex components. These component information can be seen via \textbf{Component Editor} under \textbf{general Editors} of \textbf{Window}, see figure~\ref{fig:view component information}.

\begin{figure}[h]
	\centering
	\includegraphics[width=0.5\linewidth]{"figures/View component information"}
	\caption{Vew component information}\label{fig:view component information}
\end{figure}

\subsection{Invert section}
In order to invert the selection to pick unselected faces, hold down \textbf{shift} and mass select the object...\textbf{shift} is a toggle select thus selecting the inverse.

\subsection{Deleted dangling vertices}
Delete dangling vertices, which mean there are no edges connecting to these vertices, select the whole object and press delete.

\subsection{Merge mass vertices}
Select the object and use \textbf{Edit Mesh -\textgreater Merge Components}.

\subsection{Symmetric selection}
If the \textbf{Symmetry} option is set to be \textbf{Object}, then whenever you select a element (vertex, face), the symmetric element will also be selected at the same time! To turn it off, go to \textbf{Select tool settings} and under \textbf{Symmetry Settings tab}, change the option to \textbf{Off}. Done!

\begin{figure}[h]
	\centering
	\includegraphics[width = 0.5\textwidth]{figures/symmetric_selection}
	\caption{Symmetric selection} \label{fig: Symmetric selection}
\end{figure}


\section{Modeling Toolkit}
To show or hide \textbf{Modeling Toolkit} click the first icon of \includegraphics{"figures/Modeling Toolkit"}.

\subsection{Multi-Cut Tool}
Tools to cut, slice, and insert edges on polygons.

\subsubsection{Slice a mesh}
See figure \ref{fig: slice a mesh}.
\begin{figure}[h]
	\centering
	\includegraphics[width=0.5\textwidth]{"figures/slice a mesh"}
	\caption{slice a mesh}\label{fig: slice a mesh}
\end{figure}

\subsubsection{Cut faces}
See figure \ref{fig: cut faces}.
\begin{figure}[h]
	\centering
	\includegraphics[width=0.5\textwidth]{"figures/cut faces"}
	\caption{cut faces}\label{fig: cut faces}
\end{figure}


\subsubsection{Slice a mesh with the Multi-cut tool}
You can use the Multi-cut tool to slice polygons along a plane, and then extract or delete faces above or bellow the cut.

To slice a mesh using a custom plane:
\begin{enumerate}
	\item Select the mesh you want to slice.
	\item Open the Multi-cut Tool.
	\item Click on either side of your mesh to define two slice points.\\
	An orange slice preview line appears between the slice points. This line represents the slice plane.
	\item edit your slice plane by doing the following:
	\begin{itemize}
		\item \includegraphics[scale=0.6]{"figures/reposition using mouse"}-drag to reposition the plane.
		\item Drag to reposition slice points.
	\end{itemize}
	\item (Optional) In the \textbf{Modeling Toolkit window}, select \textbf{Delete faces or extract Faces} in the \textbf{Slice Tool options}. A dotted line perpendicular to the slice preview appears, indicating the side of the mesh that will be deleted or extracted. see figure \ref{fig: multi-cut slice line}.
	
	\begin{figure}[h]
		\centering
		\includegraphics[width=0.5\textwidth]{"figures/multi-cut slice line"}
		\caption{multi-cut slice line}\label{fig: multi-cut slice line}
	\end{figure}
	
	\item Press \textbf{Enter} or \includegraphics[scale=0.6]{"figures/cut plane using mouse"}-click to cut the mesh along the slice plane.
	
	\begin{center}
		\includegraphics[width=\textwidth]{"figures/multi-cut slice"}
	\end{center}
	
\end{enumerate}



\subsection{Bridge Tool}
Creates bridging faces between selected edges or faces.

Click \textbf{Bridge} button to toggle it on and multi-select polygon components. Or holding down \textbf{Shift Key} to select polygon components then click \textbf{Bridge} button. See figure \ref{fig: Bridge Tool}.
\begin{figure}[h]
	\centering
	\includegraphics[width=0.5\textwidth]{"figures/Bridge Tool"}
	\caption{Bridge Tool}\label{fig: Bridge Tool}
\end{figure}

\subsection{Connect Tool}
Connects the selected polygon components by inserting an edge between them.

Click \textbf{Connect} button to toggle it on and multi-select polygon components. Or holding down \textbf{Shift Key} to select polygon components then click \textbf{Connect} button. See figure \ref{fig: Connect Tool}.
\begin{figure}[h]
	\centering
	\includegraphics[width=0.5\textwidth]{"figures/Connect Tool"}
	\caption{Connect Tool}\label{fig: Connect Tool}
\end{figure}

\section{Differences between Extrude and Scale}
Sometimes geometry is transformed. It can be done via \textcolor{red}{Extrude} or \textcolor{red}{Scale}. The difference between these two methods is \textcolor{red}{Extrude changes the geometry}. On the other hand, \textcolor{red}{Scale keeps the geometry ratio}.

\section{Import an image plane}
In order to import an image, an image plane should be created firstly. Click \textcolor{red}{View} in the view port panel and then click \textcolor{red}{Camera Attribute editor}. In the \textcolor{red}{Attribute editor}, find the \textcolor{red}{Environment} section and press \textcolor{red}{create}. After that select an image file to create a background image.\\

Another way is to click \textcolor{red}{Free Image plane} inside \textcolor{red}{Create Menu}.

\section{X-Ray}
\textcolor{red}{X-ray} is quite useful in viewing interior geometry or interactions between two objects. To turn on this function, follow \textcolor{red}{each view port-\textgreater Shading-\textgreater X-Ray}.

\section{make shull geometry of surfaces}
In order to create a thin shull of original surfaces, \textcolor{red}{Extrude} is required. The trick lies in after selecting the object, go to the \textcolor{red}{Edit Mesh for Face-\textgreater Extrude}. In the pop-up menu, input the \textcolor{red}{Thickness} you need and then apply.

\begin{figure}[tbh]
	\centering
	\includegraphics[width=0.5\linewidth]{"figures/Making shull by Extrude"}
	\caption{Finding distance in Maya}
	\label{fig:Making shell by Extrude}
\end{figure}

\section{Finding distances in Maya}
Go to \textbf{Create-\textgreater measure tools-\textgreater distance tool}. Now click anywhere on the grid twice to define two distance points.\\

When you're done, it will create two locators and the distance node. The distance node displays the distance of Maya units in the view port.

\begin{figure}[tbh]
	\centering
	\includegraphics[width=0.5\linewidth]{"figures/Finding distances in maya"}
	\caption{Finding distance in Maya}
	\label{fig:Finding distances in maya}
\end{figure}

\section{Clear holes}
\textcolor{red}{Fill hole} can be done simply by click \textcolor{red}{fill hole} of \textcolor{red}{Mesh menu}. A more delicate way is to select all these edges around the hole and apply the \textcolor{red}{Fill hole} command.

\section{Clear intersections}
Previously, I thought intersections can be solved simply by \textcolor{red}{deleting intersected polygons} and then \textcolor{red}{fill hole}. Thanks to \textbf{Ezhaz}, he told me that \textcolor{red}{\textit{intersections should be operated simply by deleting and filling hole due to topology. If an important vertex is eliminated, the whole model may collapse. It's like a neck connecting head and body.}}. The solution to this problem, according to \textcolor{red}{Ezhaz}, is to \textcolor{red}{\textit{manually edit vertices or edges, like delete edges and vertices while keep faces or merge some very close vertices}}. The merge option needs two vertices to be close enough. so he used \textcolor{red}{Snap to points} and \textcolor{red}{Move operations} to successfully merge two vertices into one, thus solve this problem without changing the topology.

\section{Clear Non-manifold elements}
While modelling, due to deleting vertices or edges or extrude, some non-manifold elements will be created. All these elements perform badly during post-process and tetrahedron. It is wise to check and clear these elements all the time to make sure the mesh in good condition.\\

The best way to deal with \textbf{Non-manifolds} is almost the same with dealing with \textbf{Clear intersections}. Besides \textbf{moving and merging vertices}, one more thing has to be done is to \textbf{add more edges} to clear non-manifolds.\\

\section{Blueprint modeling}
Use three cameras to capture \textbf{Front, Side and Top} views of a model, then position these photos in corresponding places in Maya and use them to build model manually. See figure \ref{fig: Blueprint modeling}.

\begin{figure}[h]
	\centering
	\includegraphics[width=\textwidth]{"figures/Blueprint modeling"}
	\caption{Blueprint modeling}\label{fig: Blueprint modeling}
\end{figure}

Another related technique is using \textbf{3D Scanner}. The principle is shown via figures:
\begin{figure}[h]
	\centering
	\begin{subfigure}[b]{0.45\textwidth} % b means alignment at the bottom
		\centering
		\includegraphics[width=\textwidth]{figures/3dscanner1}
		\caption{3dscanner1}\label{fig:subfig:3dscanner1}
	\end{subfigure}
	\begin{subfigure}[b]{0.45\textwidth}
		\centering
		%	\includegraphics[width=\textwidth]{3dscanner2}
		\caption{3dscanner2}\label{fig:subfig:3dscanner2}
	\end{subfigure}
	\begin{subfigure}[b]{0.45\textwidth} % b means alignment at the bottom
		\centering
		%	\includegraphics[width=\textwidth]{3dscanner3}
		\caption{3dscanner3}\label{fig:subfig:3dscanner3}
	\end{subfigure}
	\begin{subfigure}[b]{0.45\textwidth}
		\centering
		%\includegraphics[width=\textwidth]{3dscanner4}
		\caption{3dscanner4}\label{fig:subfig:3dscanner4}
	\end{subfigure}
	\caption{3D scanner}
	\vspace{\baselineskip}
\end{figure}

\section{Smooth rendering}
I tried \textbf{Maya Smooth}, \textbf{Mental ray Smooth} and \textbf{Soften Edge}. Personally, I prefer \textbf{Soften Edge}, as \textbf{Maya Smooth} is always to smooth all mesh regardless of what I have selected, and \textbf{Mental ray Smooth} is hard to control the light and the result. Damn it, \textbf{Mental ray} rendering can use \textbf{Soften Edge} as well.

\subsection{Soften Edge}
Select all edges will be soften and go to \textbf{Normals}, apply \textbf{Soften Edge}, then selected edges are softened! Bravo!

\subsection{Partly soften Edge}
Sometimes, I need to keep sharp corners and borders sharp so, select these edges or \textcolor{red}{convert from face selections to edges} and \textcolor{red}{Inverse} selection to select all edges need to be softened.

\subsection{Maya Smooth}
The smooth method is under \textbf{Mesh}. Number of subdivisions and method can also be set here. But more subdivisions slower the computation is. I can't bear with it. Meanwhile, I cant apply \textbf{Maya Smooth} to selected faces, what a shame!

\subsection{Mental ray Smooth}
\textbf{Mental ray Smooth} seems to be a smooth method taking effect during real time rendering. It's faster than \textbf{Maya Smooth} but the result seems almost the same as it does smooth the whole mesh. by the way, it is hard to cope with.

The \textbf{Mental ray Smooth} Editor is under \textbf{Window $ \rightarrow $ Rendering Editors $ \rightarrow $ Mental ray $ \rightarrow $ Approximation Editor}. Then select Object and create subdivisions and assign it to the object.

\section{User defined Shelf}
\begin{figure}[h!]
	\centering
	\includegraphics[width=0.7\linewidth]{"figures/User-defined shelf"}
	\caption{User defined shelf}
	\label{fig:user-defined-shelf}
\end{figure}

\section{Wireframe rendering}
Use Maya Vector to render edges.

Edge style:
\begin{itemize}
	\item Outlines: can render Quad mesh.
	\item Entire Mesh: if is a quad mesh will be converted to tri mesh implicitly.
\end{itemize}

\section{Curving editing}
To add more points and smooth curve, go to \textbf{edit curves -- rebuild curve} and change \textbf{number of spans} to wanted values.

\begin{figure}[!h]
	\centering
	\includegraphics[width=0.5\linewidth]{figures/rebuild_curve_menu}
	\caption{ rebuild curve}
	\label{fig:rebuildcurvemenu}
\end{figure}

\begin{figure}[!h]
	\centering
	\includegraphics[width=0.5\linewidth]{figures/rebuild_curve_option}
	\caption{change number of spans}
	\label{fig:rebuildcurveoption}
\end{figure}

\section{Quad draw}

\section{z up}
\begin{figure}
	\centering
	\includegraphics[width=0.7\linewidth]{figures/z_up/maya_z_up}
	\caption{z up}
	\label{fig:mayazup}
\end{figure}