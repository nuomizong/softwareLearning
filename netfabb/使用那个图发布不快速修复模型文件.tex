% !Mode:: "TeX:UTF-8"	% read in as utf8 file.
\documentclass[10pt,a4paper]{article}

% !Mode:: "TeX:UTF-8"	% read in as utf8 file.
\usepackage{microtype}

% ---- Math packages. ----
\usepackage{amsmath}
\usepackage{amsfonts}
\usepackage{amssymb}
\usepackage{amsthm}	% thm = theorem
% \usepackage{mathtools}
% \usepackage{array}

\usepackage{siunitx} % standard unit
\DeclareSIUnit\rotation{r}

% ---- Figures and Captions. ----
\usepackage{graphicx}


\usepackage{caption}
% captionof command. when inserting graphics without figure environment, captionof can produce the caption.

\usepackage{subcaption} % subfigure environment.

% ---- Tables. ----
\usepackage{booktabs} % three-line tables: toprule, midrule, bottomrule
\usepackage{longtable}
\usepackage{multirow}

% ---- References. ----
\usepackage[square,sort,comma,numbers]{natbib}
%\usepackage{biblatex}

\usepackage{hyperref} % url command

%======================================================
%	Color
%======================================================
\usepackage{color}
\usepackage{colortbl}
\definecolor{bkg}{rgb}{0.95,0.95,0.92}

%======================================================
%	Todo
%======================================================
\usepackage{todonotes}
\newcommand{\TODO}[1]{{\color{red}{[TODO: #1]}}}


\RequirePackage[l2tabu, orthodox]{nag}

%\usepackage{tikz}

\usepackage[width=21.00cm, height=29.70cm, left=2.54cm, right=2.54cm, top=2.54cm, bottom=2.54cm]{geometry}

\usepackage[utf8]{inputenc}

\usepackage{verbatim}
\usepackage{listings} % can print various codes including listings itself and LaTeX. can also use lstinline command.
% ====== set styles for listings. ======
\definecolor{codegreen}{rgb}{0,0.6,0}
\definecolor{codegray}{rgb}{0.5,0.5,0.5}
\definecolor{codepurple}{rgb}{0.58,0,0.82}

\lstdefinestyle{zstyle}{
	backgroundcolor=\color[rgb]{0.95,0.95,0.92},
	commentstyle=\color{codegreen},
	basicstyle=\ttfamily\small,
	keywordstyle=\color{codepurple},
	numberstyle=\tiny\color{codegray},
	numbersep=5pt,
	stringstyle=\color{red},
	showspaces=false,
	showstringspaces=false,
	showtabs=false,
	numbers=left,
	prebreak=\raisebox{0ex}[0ex][0ex]{\ensuremath{\hookleftarrow}},
	captionpos=b,
	frame=single,
	breakatwhitespace=false,
	breaklines=true,
	keepspaces=true,
	tabsize=4,
	%escapeinside,
}

\lstset{style=zstyle}
% ====== set styles for listings. ======

\usepackage{tabularx} % tabularx environment. equivalent lenth?

\usepackage{CJKutf8} % Chinese, Japanese, Korean input with utf-8 encoding. it loads \usepackage[utf8]{inputenc} internally



%\usepackage{fancyvrb} % not familiar with

\usepackage{cleveref}
%% \crefname{ <type> }{ <singular> }{ <plural> }
%% \Cref{ key} capitalize the first letter.
%\crefname{table}{table}{tables} 
%\crefname{figure}{fig.}{figs.}
%\crefname{equation}{eq.}{eqs.}

\usepackage{newverbs} % if listings package doesn't work, use this one to highlight.
\newverbcommand{\cverb}{\color{red}}{} % colored vertb with red
\newverbcommand{\bverb}	% verbatim with gray background
{\begin{lrbox}{\verbbox}}
	{\end{lrbox}\colorbox{gray}{\box\verbbox}}

%======================== separator

\usepackage{txfonts} % piup
\usepackage{xfrac} % more beautiful and standard fractions. \sfrac{1}{2}

%\usepackage{enumerate}

\numberwithin{equation}{section} % number equation with section number.

%\usepackage[numbers,sort]{natbib} % [1,3,2] => [1,2,3]
%\usepackage[numbers,sort & compress]{natbib} % [1,3,2] => [1-3]

%\setlength{\parindent}{0pt}

\usepackage{enumerate} % [i], [ii]...

\begin{document}
\begin{CJK*}{UTF8}{song} % CJK*环境

\author{nuomizong}
\title{使用netfabb快速修复模型文件}
\date{May 17th, 2016}
\maketitle

\tableofcontents

\newpage
如果你尝试过3D打印,你可能知道3D打印对于模型文件是有特殊要求的。在建模软件中看上去非常完美的模型,在打印时可能会遇到问题,怎么样事先预知并且解决这些问题呢?netfabb Studio Basic 可以帮助我们。netfabb 是款常用的模型编辑软件,可以测量、修复以及检查模型文件。我们熟悉的 Shapeways,也是使用 netfabb 提供的服务,对用户上传的模型文件进行检查。

这个例子中我们将使用 MeshLab 自带的海螺模型,MeshLab 是另一款常用的模型编辑软件,我们今后还会用到它。你可以在 MeshLab 安装目录中 samples 下找到 seashell.gts 这个文件。这个文件有小问题,我们来尝试修复它把。一般,快速修复包含下列几步,

\begin{itemize}
\item 在 netfabb 中加载模型并预检
\item 进行标准检查
\item 自动修正
\item 采用修正结果并再次检查
\item 输出修正结果
\item aa
\end{itemize}

让我们开始吧。

\section{加载模型并预检}
直接把模型文件拖拽到netfabb 窗口中就能加载该模型。netfabb 在加载模型后,会自动对模型进行一系列检查。主要包括模型是否有未闭合空间,是否存在相反的法线,是否有孤立的边线等。如果发现问题,会在屏幕右下角显示感叹号。正如我们在下图中看到的。

%\begin{figure}[h]
%\centering
\includegraphics[width=0.7\linewidth]{importModels}
%\caption{importModels}
%\label{fig:importModels}
%\end{figure}

如果加载模型后你没有看到红色感叹号,那么恭喜你,你的模型没有问题!

\section{进行标准检查}
在预检之后,让我们对模型进行一下更彻底的检查吧。
我们从菜单中选择 Extras $ > $ New analysis $ > $ Standard analysis

%\begin{figure}[h]
%\centering
\includegraphics[width=0.7\linewidth]{standardAnalysis}
%\caption{standard analysis}
%\label{fig:standardAnalysis}
%\end{figure}

检查结束后我们可以在右上角看到多了一个名为 “Part analysis” 的层,在画面中部,我们可以看到一个红色的 “No”,告诉我们该模型未关闭。模型未关闭的原因有很多,可能是因为模型中有没有吻合的边,或者面没有完全衔接上。这些问题可能在画面上看不出来,但是打印时会给3D打印机带来麻烦。除了这个 “No” 之外,我们看到还有个绿色的“Yes”,这表示我们的模型中不包含相反的法线,算是件好事。

\section{自动修正}
我们可以在菜单中选择 Extra $ > $ Repair parts 进行修正。在画面的右下方我们可以当前模型的一些统计信息,让我们直接点击 Automatic repair。

%\begin{figure}[h]
%\centering
\includegraphics[width=0.7\linewidth]{automaticRepair}
%\caption{automatic repair}
%\label{fig:automaticRepair}
%\end{figure}

在弹出的对话框中选择 Default repair。然后执行。

%\begin{figure}[h]
%\centering
\includegraphics[width=0.7\linewidth]{defaultRepair}
%\caption{default repair}
%\label{fig:defaultRepair}
%\end{figure}

我们可以看到模型的一些参数有了变化,最主要的 Holes 现在变为 0 了。

%\begin{figure}[h]
%\centering
\includegraphics[width=0.7\linewidth]{noHoles}
%\caption{no holes}
%\label{fig:noHoles}
%\end{figure}

\section{采用修正结果并再次检查}
让我们点击右下角的 Apply repair。这个操作会移除之前添加的 Part analysis 和 Part repair 层,并且将修正的结果应用于我们原来的模型。

%\begin{figure}[h]
%\centering
\includegraphics[width=0.7\linewidth]{repairResult}
%\caption{repair result}
%\label{fig:repairResult}
%\end{figure}

保险起见我们可以再次运行标准检查,这次我们可以看到两个绿色的“Yes”了。

%\begin{figure}[h]
%\centering
\includegraphics[width=0.7\linewidth]{doubleCheck}
%\caption{double check}
%\label{fig:doubleCheck}
%\end{figure}

\section{输出修正结果}
需要注意的时,直到此时我们都未改动原有的文件,只是在原有模型的基础上建立了新的模型。所以如果我们觉得修复效果不错的话,需要将修复结果另外导出。你可以在 Part $ > $ Export part 菜单中选择你想要导出的模型文件类型。

%\begin{figure}[h]
%\centering
\includegraphics[width=0.7\linewidth]{exportResult}
%\caption{export result}
%\label{fig:exportResult}
%\end{figure}

有时候,在你导出模型时可能会遇到对话框告知模型仍有错误。

%\begin{figure}[h]
%\centering
\includegraphics[width=0.7\linewidth]{exportRepair}
%\caption{export repair}
%\label{fig:exportRepair}
%\end{figure}

这是因为你将要导出的文件类型有着更严格的检查,此时你可以点击对话框中的 Repair 按钮,让 netfabb 针对该种文件类型进行额外的修复,通常这都能解决问题。

%\begin{figure}[h]
%\centering
\includegraphics[width=0.7\linewidth]{exportDone}
%\caption{export done}
%\label{fig:exportDone}
%\end{figure}

导出模型后,你就能进行下一步的打印。
\newpage

\end{CJK*}
\end{document}